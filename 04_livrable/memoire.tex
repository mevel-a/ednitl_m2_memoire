\documentclass[12pt, a4paper]{article}
\usepackage[utf8]{inputenc}
\usepackage[T1]{fontenc}
\usepackage{lmodern}
\usepackage[dvipsnames]{xcolor}
\usepackage{fancyhdr}
\usepackage{reledmac}
\usepackage{float}
\usepackage{graphicx}
\usepackage{wallpaper}
\usepackage{csquotes}
\usepackage[top=2.5cm, bottom=2cm, left=2cm, right=2cm, heightrounded, marginparwidth=3.5cm, marginparsep=0.3cm]{geometry}
\renewcommand{\headrulewidth}{0.1pt}
\renewcommand{\footrulewidth}{0.3pt}
\usepackage{hyperref}
\pagestyle{fancy}
\lhead{{\scshape{Mével}} Adrien M2 EdNITL}
\chead{}
\rhead{2022-2023}
\fancyfoot[]{}
\rfoot{\thepage}
\usepackage[french]{babel}
\setlength{\headheight}{20.61049pt}

%POUR PASSAGE EN REPORT SI BESOIN
\fancypagestyle{plain}{%
\fancyhf{}% clear all header and footer fields
\fancyfoot[R]{\thepage}
\renewcommand{\headrulewidth}{0pt}%
\renewcommand{\footrulewidth}{0.3pt}%
}

\begin{document}
\ULCornerWallPaper{0.23}{img/logo_u_lille.png}


\begin{titlepage}
  

\vspace*{3cm}

 
\begin{center}
\textsc{\huge Mémoire de recherche}

\textsc{\textit{Pour un nouveau roman}, édition numérique}



%Version du 
\today


\vspace*{2cm}
Mme~Florence~\textsc{de Chalonge}


M.~Matthieu~\textsc{Marchal}




\vspace*{11cm}
\small
\textsc{Mével}~Adrien

Master~2 Lettres Modernes,

«~Éditions numériques et imprimées de textes littéraires~»

\vspace*{2.5cm}
Année universitaire 2022-2023




\end{center}


\end{titlepage} 


\newcommand{\punr}{\textit{Pour un nouveau roman}}
\newcommand{\robbe}{Alain~Robbe-Grillet}
\newcommand{\galia}{Galia~\textsc{Yanoshevsky}}
\newcommand{\op}{\textit{Op. cit.}, p.~}
\newcommand{\fullgalia}{\textsc{Yanoshevsky} Galia, \textit{Les discours du Nouveau Roman~: Essais, entretiens, débats}, Villeneuve d'Ascq, Presses universitaires du Septentrion, 2006, p.~}

\vspace*{3cm}

\addcontentsline{toc}{section}{Introduction}

Constitué par un regroupement de textes publiés au fil des années selon des modalités différentes (tantôt dans des revues spécialisées tantôt dans des journaux généralistes) pour des projets (polémiques ou théoriques) particuliers, \punr{} est une œuvre composite. Il est constitué de critiques d'ouvrages, pas toujours contemporains, de textes de théorie littéraire voire même de métaphysique ou de phénoménologie~; cet ouvrage dont on ne perçoit pas toujours l'unité est pourtant loin d'un recueil dans lequel chaque article pourrait se lire indépendamment des autres, réunis par commodité plus que nécessité en un livre unique permettant au grand public de s'approprier les nouveautés de la recherche. En effet quelques fils directeurs traversent et animent les textes~: on y trouve des références spécifiques, témoignage d'une époque littéraire, un style polémique, une pensée somme toute entière et constituée~; l'œuvre tient par un effort, au demeurant moindre et \textit{a posteriori}, d'unification via l'insertion de seuil et les réécritures partielles.

Des récriminations contre «~les critiques~» à un air du temps phénoménologique critique de la métaphysique classique en passant par une certaine vision du bon-sens, l'œuvre qui se présente comme n'étant pas une théorie du roman semble pourtant (re)tracer une voie à la littérature de son époque. \punr, qu'est-ce~? Les propos épars d'un relativement jeune auteur tentant d'occuper le terrain et de promouvoir ses textes~? Un moment de Robbe-Grillet~? Une époque de l'histoire littéraire~? Un livret de recommandations (aux publics, aux critiques, aux auteurs)~? Ou bel et bien une doctrine littéraire~?

Si le texte peut être lu comme tout cela à la fois, cela tient sans doute à la complexité d'une pensée faite de recoins. Tentant de maintenir un équilibre délicat entre légitimation par la tradition et refus d'une autre tradition, entre l'affirmation de ce qu'il ne faudrait pas écrire et le refus de délivrer les consignes de la nouvelle école, \punr{} réfute et emprunte (explicitement ou non) aux modèles et aux adversaires que son auteur s'est choisis. L'habileté tenant à un art consommé de l'ironie et du sous-entendu permettant d'emprunter à ceux-là même que l'on entend discréditer.

Déclarant la pleine liberté de l'auteur tout en assénant qu'il ne faudrait plus écrire comme ceci ou comme cela, quelle unité \punr~? Nous pensons que la pleine compréhension de la manière dont diverses lectures s'imbriquent en un recueil tient à la tenue des fils polémiques et théoriques qui sous-tendent l'ensemble. 

Nous détaillerons la structure de l'œuvre, détaillant les publications initiales à l'origine de chaque article composant le recueil, avant de nous intéresser aux techniques stylistiques qui sous-tendent la rhétorique polémique de \robbe, enfin nous nous pencherons sur les théories esthétiques développées dans l'ouvrage.

\newpage

\section{La mise en recueil d'articles disparates}


%brève présentation  de la structure générale
%intro à chacun des textes

Le recueil \punr est une collection de plusieurs articles publiés dans des journaux et des revues aux profils divers, allant du quotidien national (\textit{L'Express} à la revue spécialisée \textit{Critique}. Le travail d'identification des sources primaires des articles et de la mise en recueil ayant déjà été mené par Mme~\galia\footnote{\textsc{Yanoshevsky} Galia, \textit{Les Discours du Nouveau Roman~: Essais, entretiens, débats}, Villeneuve d'Ascq, Presses universitaires du Septentrion, 2006}, cette section du présent travail reprend pour la majeure partie les observations de \galia.
\begin{quote}
    L’ensemble des essais manifeste un développement qui se donne comme un palimpseste, tout en exposant graduellement le programme qu’il propose pour le Nouveau Roman. L’effet de palimpseste est produit par la répétition des mêmes thèmes~–~profondeur et surfaces, descriptions et notions périmées tels le personnage et l’intrigue~–~chaque fois traités sous différents angles\footnote{\fullgalia87}.
\end{quote}
Le recueil est constitué de treize~articles parmi lesquels cinq sont en réalité des critiques d'œuvres, dont la portée théorique plus générale n'est explicitée que par la mise en recueil elle-même.

Au moment de la publication de \punr, \robbe{} est un auteur connu du public~: ses chroniques dans des journaux grand public (\textit{L'Express} dès~1955, \textit{Le Figaro} en~1962 et en~1963 en réponse à un critique du journal, \textit{France Observateur} tout au long de l'année~1957) donne à Robbe-Grillet une stature aux yeux du public\footnote{\fullgalia77} que la publication du recueil consacre définitivement. Encore aujourd'hui, le nom de Robbe-Grillet est synonyme de «~chef de file du nouveau roman~».

Bruce~Morissette\footnote{\textsc{Morissette} Bruce, \textit{Les Romans de Robbe-Grillet}, Paris, Éditions de Minuit, 1971 [1963]} souligne que les premières théories de Robbe-Grillet sont publiées dans les essais critiques qu'il publie dans la revue \textit{Critique} et \textit{Nouvelle Revue Française} dès 1953. Ces cinq~critiques (un sixième texte de ce sous-ensemble en constitue une introduction rédigée pour le recueil) en plus de servir à donner une filiation au mouvement du Nouveau Roman, servent également la constitution d'un public, d'une communauté de sens et de références. Lisant ces critiques, le lecteur chausse en quelque sorte les lunettes de Robbe-Grillet pour lire à travers elle. Par ailleurs, ce détour par la critique sert la mise en scène d'une pensée qui ne se veut pas purement théorique et une mise en perspective de cette pensée~: elle dépasse très largement les œuvres de Robbe-Grillet seule et permet de saisir des enjeux esthétiques contemporains aux origines plus anciennes. Enfin, le recueil, devenant par l'adjonction de ses critiques un texte ouvertement composite, paraît traiter de sujets d'autant plus riches et vastes que son contenu formel en est hétérogène.

Les critiques intégrés à \punr{} sont identifiées par \galia~:
\begin{itemize}
    \item «~Samuel Beckett, Auteur dramatique~», \textit{Critique}, n°~69, février~1953, p.~108-114
    \item «~Joë Bousquet le rêveur~», \textit{Critique}, n°~77, octobre~1953, p.~819-829
    \item «Un roman qui s'invente lui-même~»,\textit{Critique}, n°~80, janvier~1954, p.~82-88
    \item «~La conscience malade de Zeno~», \textit{Nouvelle Revue Française}, n°~19, juillet~1954, p.~138-141
    \item «~Samuel Beckett ou la présence sur la scène~», \textit{Critique}, n°~189, février~1963, p.~108-114
    \item «~Énigmes et transparences chez Raymond~Roussel~», \textit{Critique}, n°~199, décembre~1963, p.~1027-1033
\end{itemize}




Après la publication de son premier roman \textit{Les Gommes}\footnote{\textsc{Robbe-Grillet} Alain,\textit{Les Gommes}, Paris, Éditions des Minuit, 1953}. C'est après la publication de son deuxième roman \textit{Le Voyeur}\footnote{\textsc{Robbe-Grillet} Alain,\textit{Le Voyeur}, Paris, Éditions des Minuit, 1955} couronné du prix des critiques et surtout de ce que les Éditions de Minuit appelèrent «~la querelle du voyeur~», que \robbe{} commence à publier dans \textit{L'Express} des articles au visé proprement théorique dans la série de neuf~chroniques intitulée «~Littérature aujourd'hui~». D'après Bruce~Morissette, cette controverse lancée par Émile~Henriot dans \textit{Le Monde}\footnote{\textsc{Henriot} Émile, «~Le prix des critiques "\textit{Le Voyeur}", d'Alain Robbe-Grillet, \textit{Le Monde},15~juin~1955} marque le début des colloques, articles et analyses du phénomène «~nouveau roman~». Au sein de l'arène médiatique, Robbe-Grillet expose ses vues sur la littérature et s'exposer lui-même en tant que progressiste opposé, entre autre, aux Académiciens (dont Émile~Henriot fait partie). \galia{} identifie ces neufs articles\footnote{\fullgalia77}~:
\begin{itemize}
    \item «~Réalisme et révolution~», 3~janv.~1955, p.~15.%Y
    \item «~Il écrit comme Stendhal~», 25~oct.~1955, p.~8.%Y
    \item «~Pourquoi la mort du roman~?~», 8~nov.~1955, p.~8.%
    \item «~L’écrivain lui aussi doit être intelligent~», 22~nov.~1955, p.~10.%
    \item «~Les Français lisent trop~», L’Express, 6~déc.~1955, p.~11.%
    \item «~Littérature engagée, littérature réactionnaire~», 20~déc.~1955, p.~11.%Y
    \item «~Pour un réalisme de la présence~», 17~janv.~1956.%Y
    \item «~Kafka discrédité par ses descendants~», 31~janv.~1956, p.~11.%
    \item «~Le réalisme socialiste est bourgeois~», 21~févr.~1956, p.~11.%Y
\end{itemize}
Parmi lesquels seuls «~Réalisme et révolution~», «~Il écrit comme Stendhal~», «~Littérature engagée, littérature réactionnaire~», «~Pour un réalisme de la présence~» et «~Le réalisme socialiste est bourgeois~» sont intégrés à \punr. Concernant cette série d'article \galia{} note qu'elle contient déjà le ton polémique s'attaquant aux autres littérature et l'appel à un renouveau de la littérature par la fondation d'un «~nouveau réalisme~»\footnote{\fullgalia80}. Toujours dans sa thèse, \galia{} résume les thèses de l'ensemble des articles, à la lecture de ces résumés il nous paraît légitime de penser que certains articles n'ont pas été retenus car leur finalité fut jugée redondante~:
\begin{itemize}
    \item «~Pourquoi la mort du roman~?~» dénonce l'idée selon laquelle le roman serait en train de disparaître. Une nouvelle définition de ce qui constitue le réalisme suffirait permettre au roman de se renouveler et donc de survivre. En plus d'être l'un des fils directeur du recueil, cette affirmation est étayée en détail dans l'article «~Du réalisme à la réalité~» (issu d'une publication du \textit{Figaro}\footnote{\textsc{Robbe-Grillet} Alain, «~Monsieur personne répond... Pour un "nouveau roman"~», Paris, \textit{Le Figaro Littéraire}, 5-11~décembre~1963, p.~1-26}, deux articles de \textit{L'Express}\footnote{\textsc{Robbe-Grillet} Alain, «~Réalisme et révolution~», Paris, \textit{L'Express}, 3~janvier~1955, p.~15 et «~Pour un réalisme de la présence~», Paris, \textit{L'Express}, 17~janvier~1956, p.~11} et l'entretien de \textit{Tel Quel}\footnote{\textsc{Robbe-Grillet} Alain, «~La Littérature aujourd'hui~–~\textsc{vi}~», n°~14, été~1963, p.~39-45}).
    \item «~L’écrivain lui aussi doit être intelligent~» s'attaque au \textit{topos} selon lequel ce seraient l'«~alcoolisme, le malheur, la passion mystique, la folie~» des écrivains qui seraient à l'origine des chefs d'œuvre. \robbe{} défend dès le premier texte du recueil final le droit de l'écrivain à écrire et théoriser en conscience.
    \item «~Les Français lisent trop~» se veut la définition d'un public au Nouveau roman et discute la difficulté à proposer des innovations en littérature. Ce sont là les enjeux du recueil achevé régulièrement abordé, notamment dans «~Temps et description dans le roman d'aujourd'hui~».
    \item «~Kafka discrédité par ses descendants~» s'attaque à une lecture métaphysique de Kafka (selon laquelle ses romans laisserait entrevoir un «~au-delà inaccessible~»). Si \punr{} fait l'impasse sur une quelconque analyse de Kafka, les références multiples à Kafka tout au long du recueil en tant que précurseurs suffisent à imposer comme une évidence le fait que ses écrits participeraient des théories esthétiques de Robbe-Grillet.
\end{itemize}

Une seconde série d'articles est publiée au cours de l'année~1957 dans la rubrique «~La vie des lettres, Littérature d’aujourd’hui~» de l'hebdomadaire \textit{France Observateur}~:
\begin{itemize}
    \item «~Écrire pour son temps~», n°~387, 10~oct.~1957, p.~17.
    \item «~Il n’y a pas "d’avant-garde"~», n°~388, 17~oct.~1957, p.~19.
    \item «~La mort du personnage~», n°~389, 24~oct.~1957, p.~20.
    \item «~Un joli talent de conteur…~» n°~390, 31~oct.~1957, p.~19.
    \item «~La forme et le contenu~», n°~392, 14~nov.~1957, p.~19.
\end{itemize}
Identifiés par \galia, tous ces articles sont à l'origine de l'article le plus proche des thèses de Nathalie~Sarraute\footnote{\textsc{Sarraute} Nathalie, \textit{L'Ère du soupçon}, Paris, Gallimard, coll. «~folio essais~», 2019 [1956]} intitulé «~De quelques notions périmées~». Ces articles sont marqués par une opposition vis-à-vis de la critique, du moins des termes qu'elle emploie, qualifiés de «~lieux communs~» dans «~Il n’y a pas "d’avant-garde"~».

Enfin \punr{} est également constitué d'article de revues, au ton moins polémique et occupant au sein du recueil des places importantes (le premier et les derniers articles du versant plus théorique de \punr). Ces articles «~entérin[e] les conceptions développées dans les journaux\footnote{\fullgalia84}~» et consacre le passage «~d’un romancier et critique débutant et controversé, à l’expression théorique de l’écrivain qui participe activement aux polémiques littéraires de son époque, et affronte les figures principales de la scène littéraire dominante\footnote{\fullgalia84}~». Quatre articles publiés dans des revues spécialisées sont intégrés à \punr~:
\begin{itemize}
    \item «~Une voie pour le roman futur~», \textit{Nouvelle Revue Française}, n°~43, juillet~1956, p.~77-84.
    \item «~Nature, Humanisme, Tragédie~», \textit{Nouvelle Revue Française}, n°~70, octobre~1958, p.~580-603.
    \item «~Nouveau Roman, homme nouveau~», \textit{Revue de Paris}, n°~68, septembre~1961, p.~115-121.
    \item «~La Littérature aujourd'hui~–~VI~», \textit{Tel Quel}, n°~14, été~1963, p.~39-45. Il s'agit du sixième entretien d'une série d'\textit{interview} sur la situation de la littérature, Robbe-Grillet y est alors convoqué en tant que chef de file déclaré du Nouveau Roman\footnote{\fullgalia88}.
\end{itemize}


%    Recherche d'unité et commentaire des réécritures.
    
    %la constitution de l'ensemble~: thèmes récurrents, résolutions (ou non) d'ambiguïtés au fil des textes.

%utilisation des introductions à chaque article
% et la thèse de Galia ++

\subsection{Offrir un corpus théorique au Nouveau Roman}

La mise en recueil de ses articles révèle un projet~: constituer le mouvement Nouveau Roman autour d'un ensemble théorique. Comme le signale \galia, les références à Sarraute et Barthes sont la plupart du temps implicites, même volontairement effacée lors de la mise en recueil pour mettre en valeur la portée manifestaire du recueil\footnote{\fullgalia74-75}~: la théorie du Nouveau Roman sera signée Robbe-Grillet. Il semble en effet que toutes les théories ayant nourris celles de Robbe-Grillet se trouvent absorbées pour n'en former plus qu'une, relativement cohérente.

Citons quelques exemples de ces réécritures qui effacent les dettes de Robbe-Grillet, voire constituent des plagiats identifiés par \galia{} et Bruce~\textsc{Morissette}.
%exemples Sarraute



\galia{} remarque la disparition d'un épigraphe non sourcé issu de \textit{L'Ère du soupçon} au début de l'article «~De quelques notions périmés~»\footnote{\fullgalia95}~:
\begin{quote}
    le roman que seul l'attachement obstiné à des techniques périmées fait passer pour un art mineur\footnote{\textsc{Sarraute} Nathalie, \textit{L'Ère du soupçon}, Paris, Gallimard, coll.~«~folio essai~», p.~77-78}
\end{quote}

L'article est d'ailleurs truffé de très nettes dettes à \textit{L'Ère du soupçon}, telle la page~31 «~C’est même là qu’elle [la critique] reconnaît le "vrai" romancier~: "il crée des personnages"...~» échos de «~ils [les critiques] ont beau distribuer sans compter les éloges à ceux qui savent encore, comme Balzac ou Flaubert, "camper" un héros de roman et ajouter une "inoubliable figure" aux figures inoubliables dont ont peuplé notre univers tant de maître illustres\footnote{\textsc{Sarraute} Nathalie, \textit{L'Ère du soupçon}, Paris, Gallimard, coll.~«~folio essai~», p.~59}~» et à la page suivante l'expression «~"vrai romancier"~». Si le ton général des deux essais diffèrent quelque peu~: Sarraute établit des constats sur un ton bien moins polémiques, Robbe-Grillet ne rapproche jamais Balzac de Flaubert (mais parfois Dostoïevski\footnote{\op31}). Jusqu'à l'usage des guillemets pour marquer une distance critique et ironique vis-à-vis des discours prêtés à la critique, les emprunts sont frappants. 

De même la définition du personnage balzacien par Robbe-Grillet aux pages~31-32,
\begin{quote}
    Un personnage doit avoir un nom propre, double si possible : nom de famille et prénom. Il doit avoir des parents, une hérédité. Il doit avoir une profession. S’il a des biens, cela n’en vaudra que mieux. Enfin il doit posséder un «~caractère~», un visage qui le reflète, un passé qui a modelé celui-ci et celui-là. Son caractère dicte ses actions, le fait réagir de façon déterminée à chaque événement. Son caractère permet au lecteur de le juger, de l’aimer, de le haïr. C’est grâce à ce caractère qu’il léguera un jour son nom à un type humain, qui attendait, dirait-on, la consécration de ce baptême.
\end{quote}
semble une réécriture de Sarraute~:
\begin{quote}
    Il [le personnage] était très richement pourvu, comblé de biens de toute sorte, entouré de soins minuteux~; rien ne lui manquait, depuis les boucles d'argent de sa culotte jusqu'à la loupe veinée au bout de son nez. Il a, peu à peu, tout perdu~: ses ancêtres, sa maison soigneusement bâtie, bourrée de la cave au grenier d'objets de toute espèce, jusqu'aux plus menus colifichets, ses propriétés et ses titre de rente, ses vêtements, son corps, son visage, et, surtout, ce bien précieux entre tous, son caractère, qui n'appartenait qu'à lui, et souvent jusqu'à son nom\footnote{\textsc{Sarraute} Nathalie, \textit{L'Ère du soupçon}, Paris, Gallimard, coll.~«~folio essai~», p.~61}.  
\end{quote}

\galia{} observe également l'absence du compte rendu «~Le réalisme, la psychologie et l’avenir du roman\footnote{\textsc{Robbe-Grillet} Alain, «~Le réalisme, la psychologie et l’avenir du roman~», \textit{Critique}, n°~111-112, août-septembre~1956, p.~695-701}~» traitant de l'essai de Sarraute\footnote{\fullgalia96} puis paraphrase ce que Robbe-Grillet dira de cette absence~: «~Dans l’acte fondateur du recueil, Robbe-Grillet ne peut pas se permettre d’étaler les différences d’opinion qui se font jour au sein du mouvement naissant\footnote{\fullgalia98}~», en effet Sarraute et Robbe-Grillet partage des constats mais sans doute pas les solutions. Lorsque la première souhaite une littérature au plus près de la psychologie, le second souhaite une littérature des surfaces. 
%sarraute
%\begin{quote}
 %   En effet, dans le passage entre les articles originels et l’essai final, l’appel à l’autorité de Sarraute se fait par un jeu d’inclusion et d’exclusion de la référence explicite. Dans la version d’origine de l’essai (publiée dans la Nouvelle Revue Française), Robbe-Grillet fait appel à l’autorité de Sarraute par un épigraphe tiré de \textit{L’Ère du Soupçon}, sans pour autant noter explicitement que ce passage lui appartient. Or dans la version finale qui paraît au sein du recueil, cet épigraphe est omis. On peut dès lors se demander pourquoi l’épigraphe, type de citation utilisé pour relier un discours nouveau à un ensemble textuel plus large en l’intégrant dans l’espace des énonciations qui le précèdent, est effacé de la version finale qui figure dans Pour un nouveau roman. Le gommage du renvoi explicite à Sarraute comme source directe est sans doute dû au fait que l’inclusion de l’essai dans un recueil nouveau demande de le détacher de son appartenance à un ensemble préalable, en l’occurrence de le couper des écrits de Sarraute. Bien qu’ayant comme point de départ ou comme catalyseur les essais théoriques de cette dernière, Robbe-Grillet veut, dans la logique du nouvel ensemble, donner à ses propres essais mis en recueil une unité dotée de son autonomie propre.
    
   % Suivant la même logique, «~Le réalisme, la psychologie et l’avenir du roman~» compte rendu de L’Ère du soupçon publié initialement dans la revue Critique\footnote{\textsc{Robbe-Grillet} Alain, «~Le réalisme, la psychologie et l’avenir du roman~», \textit{Critique}, n°~111-112, août-septembre~1956, p.~695-701}, ne figurera pas dans le recueil final, quoique ce texte participe, selon Sarraute, à leur tentative commune de créer «~une sorte de mouvement~»\footnote{\fullgalia95-96}
%\end{quote}

À propos des emprunts à Barthes, \galia{} écrit~:
\begin{quote}
l’essai «~Pour un nouveau roman~» est entièrement fondé sur l’article «~Littérature objective\footnote{\textsc{Barthes}~Roland, «~Littérature objective~», Critique vol.~\textsc{xv}, n°~86-87, juillet-août~1954, p.~581-591.}~» (1954) où Barthes analyse \textit{La Jalousie} et \textit{Le Voyeur}. Non seulement Robbe-Grillet y reprend les vocables qui caractérisent le discours de Barthes, mais il emprunte également les idées et les concepts de celui-ci\footnote{\fullgalia98}.    
\end{quote}
Parmis lesquel, elle cite page~99
\begin{quote}
Des expressions comme «~des mots à caractère viscéral, analogique ou incantatoire~», «~le cœur romantique des choses~», ou encore l’usage de la métaphore des fouilles archéologiques
\end{quote}

Réunissant les principaux propos théorique sur le Nouveau Romain au sein de ses écrits, Robbe-Grillet impose son recueil comme un manifeste, hétérogène certes, mais d'un seul bloc et lui même comme le chef de file du mouvement.
% + sur Sarraute%+ pp. 108-109
% + sur sarraute%+ p. 112









%exemple Barthes

Nous pensons également que si Robbe-Grillet procède à l'effacement de certaines sources primaires pour profiter pleinement de l'autorité que lui confère le recueil et de la notoriété qu'il a acquis dans l'arène médiatique, le fait que les référents ne soient effacés qu'au moment de la mise en recueil dit peut-être quelque chose de la gestation de \punr. En effet les articles les plus inspirés de Sarraute sont aussi les plus anciens~: Robbe-Grillet n'est alors le chef de file d'aucune école mais un auteur qui défend sa pratique et la portée manifestaire qu'acquièrent plus tard ces articles aux yeux du public et de la critique n'est pas prévue initialement mais s'impose à l'auteur au fil des années. Dès lors, le gommage des références positives, pourrait être en plus d'une appropriation, un choix formel~: cela encourage l'unité du recueil davantage clôt sur lui-même, se suffisant à lui-même, et ce, d'autant plus que la pensée de Robbe-Grillet émerge d'une nécessité inscrite dans son temps et non en laboratoire. Ainsi se Robbe-Grillet se défendrait de produire une littérature de laboratoire tel à la page~145~: «~nous, au contraire, qu’on accuse d’être des théoriciens, nous ne savons pas ce que doit être un roman, un vrai roman~; nous savons seulement que le roman d’aujourd’hui sera ce que nous le ferons~».

%Par ailleurs, ces dettes théoriques ou esthétiques même si elles sont en partie gommées peuvent constituer un intertexte somme toute évident au lecteur contemporain ou actuel quelque peu lettré. Il n'empêche, il semble qu'aujourd'hui l'apport de Nathalie~Sarraute à la théorie au mouvement du Nouveau Roman, dont elle semble pourtant la première représentante véritable, a été relégué à une place moindre dans la vulgate littéraire.

Par ailleurs, ces dettes théoriques ou esthétiques même si elles sont en partie gommées, dénotent une littérature avant-gardiste en cela que ces référents techniques et esthétiques sont immédiatement contemporains et s'oppose à une vulgate datée convoquée sur le mode polémique ou parfois ironique~:
\begin{quote}
    On connaissait le théâtre d’idées. C’était un sain exercice d’intelligence, qui avait son public (bien qu’il y soit fait parfois bon marché des situations et de la progression dramatique). On s’y ennuyait un peu, mais on y «~pensait~» ferme, dans la salle comme sur la scène\footnote{\op126}.
\end{quote}
Le calembour (s'ennuyer ferme) nous paraît faire référence à la théorie des trois~publics du théâtre de Victor~Hugo~:
\begin{quote}
    la foule est tellement amoureuse de l'action qu'au besoin elel fait bon marché des caractères et des passions. Les femmes, que l'action intéresse d'ailleurs, sont si absorbées par les développements de la passion, qu'elles se préoccupent peu du dessin des caractères~; quant aux penseurs, ils ont un tel goût de voir des caractères, c'est-à-dire des hommes, vivre la scène, que, tout en accueillant volontiers la passion comme incident naturel dans l'œuvre dramatique, ils en viennent presque à y être importunés par l'action. Cela tient à ce que la foule demande surtout au théâtre des sensations~; la femme, des émotions~; le penseur, des méditations. Tous veulent un plaisir~; mais ceux-ci, le plaisir des yeux~; celles-là, le plaisir du cœur~; les derniers, le plaisir de l'esprit\footnote{\textsc{Hugo}~Victor, \textit{Ruy Blas}, Paris, Le Livre de proche, 2009 [1839], p.~15-16}.
\end{quote}
Sans doute voit-on là un éventuel référent des critiques académiques ou de tous ceux dont «~les préoccupations d’écrivains datent de plusieurs siècles\footnote{\op18}~». Par opposition à Robbe-Grillet dont les références théoriques sont contemporaines, voire même inspirées de sa seule pratique et du bon sens (voir \textit{infra}~: \ref{bonsens}), les adversaires de Robbe-Grillet ont des préoccupations datées et pratiquent un théâtre trop intellectuel où l'on s'ennuie. Ce «~théâtre d'idée~» pourrait au demeurant être celui de Sartre qui a déjà créé la plupart de ces pièces lorsque paraît l'article sur Beckett dont est extrait ce passage.



\subsection{Une filiation au nouveau roman}

À la recherche d'une légitimité refusée par les critiques traditionnels, Robbe-Grillet s'emploie à fonder une lecture de la tradition littéraire. Il y aurait d'un côté Balzac et une tradition dont la nature exacte n'est jamais caractérisée par autre chose que son rejet du Nouveau Roman et de l'autre une quête ancienne de la modernité, qui serait le propre de la valeur littéraire.
%à la recherche de l'origine de la modernité contemporaine.




Sans doute, point crucial de l'œuvre~: donner plus qu'une légitimité au Nouveau Roman, une nécessité~: le Nouveau Roman est l'aboutissement d'évolutions anciennes, soit l'avenir du roman des années~1950. En effet, semble effleurer le texte la possibilité d'un progrès en art, du moins un sens à l'histoire littéraire.
L'ordre dans lequel les «~éléments d'une anthologie moderne~» sont composés, nous paraît à ce titre éclairant. Les critiques n'y sont en effet pas organisées en fonction de l'ordre de publication des articles originaux mais bien allant du moins au plus moderne. Une gradation semble sous-entendu même au lecteur néophyte par l'éloignement de deux relatifs inconnus datés (Raymond~Roussel et Joë~Bousquet) et le célèbre contemporain (Samuel~Beckett). La lecture de ces chroniques ne vient pas démentir cette impression. Le premier article dit de Raymond~Roussel qu'il est «~parmi les ancêtres directs du roman moderne\footnote{\op88-89}~», qu'il «~n’a rien à dire et il le dit mal\footnote{\op88}~», le dernier n'a pas besoin de préciser que Robert~Pinget, publié chez Minuit depuis~1956, déconstruit le romanesque. On remarquera, à raison, que le dénigrement de Raymond~Roussel constitue davantage une concession faite à la critique (soit une nouvelle attaque à son encontre)~: le fait de n'avoir rien à dire fait clairement référence à Sartre\footnote{%ICIICIICIICIICIICIICIICIICIICIICIICIICIICIICIICIICIICIICIICIICIICIICIICIICIICIICIICIICIICIICIICIICIICIICIICIICIICIICIICIICIICIICIICIICIICIICIICIICIICIICIICIICIICIICIICIICIICIICIICIICIICIICIICIICIICIICIICIICIICIICIICIICIICIICIICIICIICIICIICIICIICIICIICIICIICIICIICIICIICIICIICIICIICIICIICIICIICIICIICIICIICIICIICIICIICIICIICIICIICIICIICIICIICIICIICIICIICIICIICIICIICIICIICIICIICIICIICIICIICIICIICIICIICIICIICIICIICIICIICIICIICIICIICIICIICIICIICIICIICIICIICIICIICIICIICIICIICIICIICIICIICIICIICIICIICIICIICIICIICIICIICIICIICIICIICIICIICIICIICIICIICIICIICIICIICIICIICIICIICIICIICIICIICIICIICIICIICIICIICIICIICIICIICIICIICIICIICIICIICIICIICIICIICIICIICIICIICIICIICIICIICIICIICIICIICIICIICIICIICIICIICIICIICIICIICIICIICIICIICIICIICIICIICIICIICIICIICIICIICIICIICIICIICIICIICIICIICIICIICIICIICIICIICIICIICIICIICIICIICIICIICIICIICIICIICIICIICIICIICIICIICIICIICIICIICIICIICIICIICIICIICIICIICIICIICIICIICIICIICIICIICIICIICIICIICIICIICI
} et le «~bien~» est l'une des expressions favorites des critiques rejetée par Robbe-Grillet\footnote{\op29 «~talent de conteur~», p.~51 «~Untel a quelque chose à dire et il le dit bien~»}~; cependant ce n'est vrai que de la première itération\footnote{\op87, à propos de n'avoir rien à dire et p.~88 sur le fait d'écrire mal}, la concession est ensuite réitérée sans mention d'une origine tierce (guillemets ou discours indirect libre) marquant, selon nous, le fait qu'elle est entérinée.

Une telle concession s'observe également dans l'article suivant traitant de Joë~Bousquet, cette fois-ci, sans même le recours à la médiation d'une \textit{doxa} à laquelle il faudrait répondre~:
\begin{quote}
    Même sans lui tenir rigueur de l’emploi fréquent du mot «~âme~» là où le mot \textit{imagination} conviendrait certes plus clairement à son propos, nous ne pouvons réprimer notre agacement devant l’espèce de mysticisme (d’ailleurs hérétique) qui baigne toute la pensée de Bousquet. Plus grave qu’un vocabulaire suspect («~âme~», «~salut~», etc.), il y a chez lui cette tentative de \textit{récupération} \textit{globale} de l’univers par l’esprit humain. L’idée de \textit{totalité} mène toujours plus ou moins directement à celle de vérité absolue, c’est-à-dire supérieure, donc bientôt à l’idée de Dieu\footnote{\op117}.
\end{quote}

Ces réserves s'évanouissent à mesure que l'on progresse dans l'anthologie dont le ton devient de plus en plus dithyrambique. Plus généralement mais particulièrement dans les éléments de l'anthologie moderne, une évolution où nulle place n'est laissé au regret se fait sentir dans la lecture de l'histoire littéraire que propose Robbe-Grillet.

Nous pensons donc pouvoir établir une typologie des auteurs cités comme précurseurs de la modernité ou eux-même modernes. Cette classification entend formaliser sommairement une filiation qui, si elle n'est pas tout à fait implicite, n'est jamais énoncée comme telle avec tant de clarté. Nous pensons mettre en valeur ici une axiologie structurante de la pensée de \punr, tenue par le style et l'agencement rhétorique du recueil.

\subsubsection{Les précurseurs illustres}
Il y aurait les précurseurs illustres qui ne sont que peu commentés mais convoqués afin de donner une légitimité au Nouveau Roman. On y trouve des grands noms de la littérature internationale~: Kafka, Flaubert, Proust, Dostoïevski, Faulkner et Joyce~; maintes fois cités séparément ou associés tout au long du recueil et tous réunis page~146, voire même désignés à la page suivante par la formule «~Kafka, Faulkner et tous les autres~» marquant la proximité supposée du lecteur avec ce regroupement.

À l'exception de Flaubert et de Proust, il est frappant de n'y trouver que des auteurs non francophones, notamment anglophone. Les auteurs anglophones semblan au demeurant un trope chez les nouveau romanciers, comme en témoigne un entretien de Claude~Simon réalisé 1988\footnote{\textsc{Simon} Claude et \textsc{Veinstein} Alain, \textit{La nuit sur un plateau}, France culture, émission du 8~février~1988 [En ligne~: \href{https://www.radiofrance.fr/franceculture/podcasts/les-nuits-de-france-culture/claude-simon-j-ai-appris-a-ecrire-dans-joyce-et-dans-faulkner-6832681}{https://www.radiofrance.fr/franceculture/podcasts/les-nuits-de-france-culture/claude-simon-j-ai-appris-a-ecrire-dans-joyce-et-dans-faulkner-6832681}]}.
%le trope américain, Butor et autres critiques que j'ai pu en tête trouve pu la ref :'(

Si \robbe{} se permet d'utiliser Dostoïevski comme élément comparant péjoratif à la page~31 (on lui reproche d'avoir, comme Balzac le père Goriot, créé des personnages inoubliables), tous sont convoqués pour mettre en valeur une modernité naissant qui va en s'affirmant toujours plus.

Enfin, si l'usage de Kafka et Dostoïevski n'es pas sans rappeler l'article «~De Kafka à Dostoïevski~» issu de \textit{L'Ère du soupçon}\footnote{\textsc{Sarraute} Nathalie, \textit{L'Ère du soupçon}, Paris, Gallimard, coll. «~folio essais~», 2019 [1956], p.~15-55
}, notons qu'alors que Nathalie~Sarraute tenait un discours théorique sur la proximité de ces deux auteurs présentés par les critiques de son époque comme étant les figures de deux arts romanesques opposés~; \robbe, lui, ne se sert de ces référents que comme d'autorité sous  lesquel se placer.

C'est là l'enjeux~: se réclamer d'un héritage glorieux dont le lecteur perçoit l'homogénéïté sans se risquer à le contredire afin de conférer à ses propres écrits une légitimité incontestable.

\subsubsection{Les précurseurs relativement méconnus}
Il s'agit de Raymond~Roussel, Joë~Bousquet, Italo~Svevo~; trois auteurs à propos desquels Robbe-Grillet a écrit des comptes-rendus intégrés au recueil. Nettement moins illustres mais loin d'être tout à fait inconnus ces trois auteurs semblent servir à justifier l'assertion selon laquelle, les recherches formelles du Nouveau Roman serait à considérer avec circonspection, voire mépris.

Notons l'inversion produite par la mise en recueil~: ces critiques étant généralement parmi les premiers écrits publics de \robbe, ils contribuèrent à installer sa stature d'écrivain, grâce à laquelle une fois le recueil constitué, il semble mettre à portée du grand public des auteurs connus mais encore considéré comme aussi illustres que Kafka ou Dostoïevski.




\subsubsection{Les modernes}


Catégorie qui paraît la plus difficile à délimiter clairement tant Robbe-Grillet emploie volontier l'expression «~roman moderne\footnote{\op32, p.~36, p.~38, p.~89}~» sans jamais mentionné d'auteurs contemporain qu'il y rattacherait. Plus encore que les autres cette catégorie doit être investie par le lecteur afin qu'il dégage lui-même les auteurs ou les œuvres citées qui semble devoir y être rattachés.


Gardons à l'esprit que ces catégories et particulièrement celle-ci, n'étant pas clairement définies mais seulement mentionnées de manière allusive laisse au lecteur le soin d'effectuer les recoupements. L'exercice est d'autant plus difficile que les plus clairement identifiés comme étant «~des grandes œuvres contemporaines\footnote{\op32}~»~: \textit{La Nausée}, \textit{L'Étranger}, Beckett, Faulkner, \textit{Le Château} et \textit{Voyage au bout de la nuit}, sont réunies à ce stade du texte pour ne plus être mentionnées (le \textit{Voyage}) ou âprement critiquées (\textit{La Nausée}, \textit{L'Étranger}). L'appel régulier à Beckett est ici frappant~: il fait l'objet d'une critique très élogieuse et est souvent convoqué comme le signe d'un changement d'époque. Cela semble faire de lui plus qu'un contemporain, un contemporain illustre~: Beckett est alors un auteur connu et reconnu dont nul ne remétra en cause la modernité. Si l'on pourrait dire de Pinget qu'il est montré comme tout aussi moderne (si ce n'est plus d'après l'ordre des critiques), sa notoriété n'égalant pas celle de Beckett, \robbe{} se garde de le mentionner. En quête de légitimité \robbe{} semble appliquer à ces contemporains le traitement que lui réserve (selon lui) la critique~: un «~demi-silence\footnote{\op7}~». Le fait que le Nouveau Roman n'est jamais été (et plus encore à ces débuts) un groupe clairement constitué explique sans doute en partie la difficulté de citer des contemporains si proches dont les travaux empruntent parfois des directions tout à fait différentes. Le point commun des modernes est une définition négative, pour des formes nouvelles mais surtout contre une forme traditionnelle~:
\begin{quote}
«~ceux qui cherchent de nouvelles formes romanesques, capables d’exprimer (ou de créer) de nouvelles relations entre l’homme et le monde, tous ceux qui sont décidés à inventer le roman, c’est-à-dire à inventer l’homme. Ils savent, ceux-là, que la répétition systématique des formes du passé est non seulement absurde et vaine, mais qu’elle peut même devenir nuisible : en nous fermant les yeux sur notre situation réelle dans le monde présent, elle nous empêche en fin de compte de construire le monde et l’homme de demain\footnote{\op9-10}.~»    
\end{quote}

C'est pourquoi Sartre et Camus pourtant cités parmi les grands auteurs contemporains sont également rejetés hors de la modernité au sein de l'article «~Nature, humanisme, tragédie~» car ils ne vont pas assez loin dans le rejet des formes anciennes. Ainsi lit-on pages~70-71 que \textit{L'Étranger} «~n’est-il pas écrit dans un langage aussi \textit{lavé} que les premières pages peuvent le laisser croire~» et citant Sartre\footnote{\textsc{Sartre} Jean-Paul, \textit{Situations I}, Paris, Gallimard, coll.~«~folio essais~», [1947], 2020} «~il pense que Camus, "infidèle à son principe, fait de la poésie". Ne peut-on pas dire, plutôt, que ces métaphores sont justement l’explication du livre ? Camus ne refuse pas l’anthropomorphisme, il s’en sert avec économie et subtilité, pour lui donner plus de poids~». Puis s'attaquant à \textit{La Nausée}, \robbe{} va s'employer à en soulever toutes les fautes stylistiques qui en font un ouvrage de la tragédie et de la profondeur~: «~Tout se passe donc comme si Sartre~– qui ne peut pourtant pas être accusé d’essentialisme –~avait, dans ce livre du moins, porté à leur plus haut degré les idées de nature et de tragédie. Une fois de plus, lutter contre ces idées n’a fait d’abord que leur conférer des forces nouvelles\footnote{\op76}~».
Là est l'erreur de Camus et de Sartre, ils ne sont pas assez défaits des formes classiques~: Robbe-Grillet relève les adjectifs viscéraux employés par Sartre ou Camus souligne l'analogie comme seul mode de description envisagé par le narrateur\footnote{\op75}, dit de \textit{L'Étranger} qu'il n'est pas assez «~lavé~». Ces fautes sont à considérer comme des erreurs stylistiques~: on est là dans l'antithèse exact de ce que Robbe-Grillet considère devoir être la voie du roman futur~:

\begin{quote}
    Refuser notre prétendue «~nature~» et le vocabulaire qui en perpétue le mythe, poser les objets comme purement extérieurs et superficiels [...]. Aussi rien ne doit-il être négligé dans l’entreprise de nettoyage. En y regardant de plus près, on s’aperçoit que les analogies anthropocentristes (mentales ou viscérales) ne doivent pas être mises seules en cause. Toutes les analogies sont aussi dangereuses\footnote{\op64}.
\end{quote}

\begin{quote}
    C’est donc tout le langage littéraire qui devrait changer, qui déjà change. Nous constatons, de jour en jour, la répugnance croissante des plus conscients devant le mot à caractère viscéral, analogique ou incantatoire. Cependant que l’adjectif optique, descriptif, celui qui se contente de mesurer, de situer, de limiter, de définir, montre probablement le chemin difficile d’un nouvel art romanesque\footnote{\op27}.
\end{quote}

Il semblerait que parmi les modernes il faille surtout considérer \robbe{}~: définissant les critères de la modernité à partir de ses œuvres et selon son style, le moindre écart éloigne de fait l'auteur de la modernité. Il convient ici de ne pas perdre de vue que cet effet vient également de la nature des textes intégrés au recueil~: ces écrits théoriques servent d'abord à défendre les œuvres de son auteur et à les rendre accessible au large public. Cela explique également pourquoi Claude~Simon est, lui, mentionné explicitement comme un auteur aussi mdoerne que Robbe-Grillet dans «~Nouveau roman, homme nouveau~» paru en~1961. Ce texte dont les considération esthétiques sont bien plus générales que les précédents est plus tardif et tente d'instaurer le collectif du Nouveau~Roman.




Fait méritant commentaire~: parmi les grandes œuvres contemporaine sont mentionnées deux~auteurs (Faulkner et Kafka) pourtant maintes fois identifiés tout au long du texte comme des œuvres du passé, proche, mais du passé. Nous pensons que cela illustre la double détermination des catégories selon les postulats de l'œuvre~: une théorie diachronique de l'évolution des formes littéraires et une théorie esthétique tendant vers la synchronie. Ainsi les auteurs début~\textsc{xx}\textsuperscript{e} peuvent être rattachés à un passé proche tout en étant convoqués comme membre du contemporain. De ces deux postulats, il faut bien saisir que le second prime sur le premier~: d'où l'exclusion si frontales des existentialistes et l'inclusion d'auteurs relativement anciens au sein de l'ensemble contemporain. La littérature du passé, notamment la tradition, semble selon Robbe-Grillet devoir être lue avec les outils du présent et au besoin, réévaluée.




%Pinget



Notons encore une fois l'absence de Nathalie~Sarraute. Sont-ce ici les désaccords sur la modernité littéraire à venir (déjà abordés \textit{supra}) ou le besoin d'effacer de trop importantes dettes théoriques~?

De manière générale nous observons que plus les référents sont lointains et/ou illustres moins ils sont cités~; plus les référents sont contemporains, moins ils sont mentionnés et si quelques-uns sont très cités c'est qu'ils sont critiquables. Il semble que chez Robbe-Grillet pour faire partie des modernes irréprochables, il vaille mieux être mort. À cette attitude, nous voyons plusieurs explications~:
\begin{itemize}
    \item Robbe-Grillet ne souhaite pas qu'un autre moderne lui fasse de l'ombre.
    \item Vanter l'écriture d'un contemporain, c'est se risquer à se dédire le jour où l'on (soi-même ou le public auquel on s'adresse) le trouvera, finalement, moins moderne.
    \item Les publications originales s'apparent davantage à des billets d'humeurs défendant la littérature de Robbe-Grillet selon des principes esthétiques généraux, avant de définir un groupe autour de ces principes dont Robbe-Grillet est l'exmple le plus proche de son auteur.
    \item En quête de légitimité, il paraît plus efficace de se réclamer d'anciens illustres que de contemporains inconnus (du moins, pas encore illustre).
\end{itemize}




\subsection{Introductions aux articles}

Afin d'offrir un aperçu des spécificités de chacun des articles et d'expliciter leur place au sein de l'économie générale du recueil, nous reproduisons ici les introductions à ces articles rédigés pour figurer dans notre édition critique numérique.
\subsubsection{À quoi servent les théories}
Premier texte du recueil, cet article est identifiée par Galia \textsc{Yanoshevsky} comme une réécriture augmentée des articles «~Il écrit comme Stendhal~» publié le 25~octobre~1955 dans \textit{L'Express} et «~La littérature, aujourd'hui - VI~» publié dans le numéro 14 de \textit{Tel Quel} en 1963. Outre sa fonction de seuil et d'introduction aux écrits théoriques qui suivent ce texte place les jalons de la rhétorique de Robbe-Grillet.

Se plaignant de la récéption de ses œuvres puis des articles publiés dans \textit{L'Express} qui sont depuis devenus le présent recueil, Robbe-Grillet se positionne comme tenant du bon sens~: il n'est pas un théoricien par vocation mais par réaction. Il ne s'agit pas tant de produire une nouvelle théorie que d'«~éclair[er] davantage les éléments qui avaient été les plus négligés par les critiques, ou les plus distordus~», combattre les «~mythes du XIXe siècle~» et de «~tente[r] de préciser quelques contours~». Ce faisant Robbe-Grillet se place en moderne (hériter d'une tradition vivante qu'il commence d'esquisser ici) en opposition aux tenants d'une tradition «~immobile, figée~» voire «~nuisible~». Contre les critiques, contre Sartre, Robbe-Grillet défend (non pas «~un~») mais l'«~exercice problématique de la littérature~» qui constitue, avec le rejet d'une écriture traditionaliste décriée comme relevant d'un pastiche éculée et avec la condamnation d'une critique psychologisante, le premier jalon théorique de l'ouvrage et une véritable définition du rôle de la littérature complétée au fil du recueil.

Notons enfin une technique usuelle chez Robbe-Grillet consistant à vider l'argumentation en limitant autant que possible la portée de ces écrits théoriques, si l'on aurait tord de considérer que Robbe-Grillet s'autorise ici à se contredire, il est juste de constater qu'il prend soin de se dégager un espace entre sa pratique littéraire et sa pratique théorique dans lequel le lecteur devrait l'autoriser à rafiner sa pensée, la préciser et sans rien renier du cheminement lui-même, en retrancher quelques passages.


\subsubsection{Une voie pour le roman futur}
Initialement publié en 1956 dans le numéros 43 de la\textit{ Nouvelle Revue Française} et enrichi des articles «~Réalisme et révolution~» et «~Pour un réalisme de la présence~» publié dans \textit{L'Express} en janvier 1955 et en janvier 1956 respectivement, cet article constitue sans doute l'une des charges les plus violentes adressés aux contemporains de Robbe-Grillet au sein du recueil.

Plus encore c'est sans doute dans ce texte que l'on mesure l'habileté (ou la mauvaise foi) de Robbe-Grillet en sa capacité à se mettre à la place de ses adversaires, qui dès lors ne peuvent que sembler des adversaires supposés, imaginaires, les personnages d'un roman.

L'auteur se montre à leur égard tantôt moqueur «~Mais tous avouent, sans voir là rien d'anormal, que leurs préoccupations d'écrivains datent de plusieurs siècles~», compréhensif «~Comment feraient-elles [les choses] pour changer ? Vers quoi iraient-elles ?~» enfin pédagogue «~Or le monde n'est ni signifiant ni absurde. Il est, tout simplement.~».

%<p>Cette variété de ton, si elle sert la démonstration, doit aussi être perçue comme les traces de l'assemblage qui produisit l'article en cette forme. [justifier voir GALIA]</p>
    
    
Au service de la démonstration cette variété de ton est la reprise d'éléments argumentatifs déjà développés semblent des traces de l'assemblage des différents articles de presses.

Au demeurant, le glissement à un ton plus théorique que polémique sert aussi l'économie globale du recueil. Par son titre et son sujet apparent : l'avenir du roman, cet article semble programmatique. Loin de délivrer des propos abstraits sur l'avenir du roman, ce texte s'avère être un état des lieux de la littérature contemporaine, tenant deux pants de la littérature contemporaine, les héritiers de Balzac, du côté des consommateurs, de ceux qui croit au «~cœur~» humain éternel et de la littérature telle qu'elle est, en opposition à la littérature telle qu'elle pourrait être, les littérature acceptant le risque de jouïr pleinement de leur liberté d'écrivain pour écrire une littérature plus proche de la réalité.

Est introduit dans ce chapitre surtout un hiatus entre l'importance de ces deux littératures et leur degré de contemporanéité. Alors que la littérature dominnante repose au mieux sur des idées de plusieurs siècle, au pire proprement délirante (le «~coeur romantique des choses~»)~; une littérature consciente du temps présent et de sa responsabilité à résister contre «~[l']appropriation systématique~» a toute les difficultés à exister.

Si ce texte semble dérouler toutes les raisons psychologiques (chez les lecteurs, les auteurs, les critiques) etc. qui empêche à cette littérature nouvelle de prendre la place qui lui est dûe, c'est bien un paradoxe qu'expose Robbe-Grillet invitant le lecteur à se positionner pour ce langage littérautre qui «~déjà change~».


\subsubsection{Sur quelques notions périmées}
Issu de cinq publications entre octobre et novembre 1957 dans le magazine \textit{France Observateur} («~Écrire pour son temps~»,«~Il n'y a pas "d'avant-garde"~»,«~La mort du personnage~»,«~Un joli talent de conteur~»,«~La forme et le contenu~») et deux publications dans \textit{L'Express} en décembre 1955 et février 1956 «~Littérature engagée, littérature réactionnaire~»,«~Le réalisme socialiste est bourgeois~», cet article plus qu'une charge envers la littérature de son temps s'attaque à la conception même que l'on se fait de la littérature s'en prenant aux termes employés pour en parler.

Commençant par développer le sens de ces termes tels que la critique les emploie, Robbe-Grillet s'ingénie à en démontrer l'inanité en s'appuyant sur des exemples qui constitue en creux une filiation dont le dernier né n'est autre que le Nouveau~Roman.

Si le texte s'emploie à démontrer que ces concepts sont «~périmés~», ce n'est pas seulement du fait d'un besoin de variété dans la production littéraire mais bien que les temps ont changé. Certaines certitudes se sont évanouies, outre les notions des critiques académiques, c'est peut-être une attitude face au monde teintée de métaphysique, de romantisme ou de positivisme, qui est ici rejettée. Plus qu'une vision nouvelle de ce que sont l'art et les œuvres, «~De quelques notions périmées~» nous dit~: la modernité c'est de ne plus croire. Cette non-croyance, ou plutôt ce soupçon, se pose non seulement sur les théories mais également sur leurs vecteurs : la forme.


\subsubsection{Nature, humanisme, tragédie}
Initialement publié en octobre~1958 dans le numéro~70 de la \textit{Nouvelle Revue Française}, ce passage nous paraît être le plus existentialiste de \textit{Pour un nouveau roman}. Si existentialiste que la philosophie de Sartre et l'absurde de Camus semblent des fétiches, du moins des œuvres qui ne prennent pas acte de ce qu'elles énoncent.

À lire ces pages, il semble que l'existentialisme prenant acte d'un monde dépourvu de sens n'a pas su se défaire de la métaphysique, entraînant un hiatus irrémédiable entre l'homme et le monde, dû à son incapacité à se défaire d'outils qu'il sait défectueux. Ce hiatus, Robbe-Grillet à la suite de Barthes propose de l'appeler «~la Tragédie~».

Lorsque l'on cessera de chercher l'homme partout on pourra peut-être s'intéresser aux phénomènes, nous dit en substance Robbe-Grillet s'attelant à donner au lecteur les clefs de la littérature qu'il est en train de produire, nous sommes presque devant un manuel, du moins des recommandations techniques pour une forme débarassée de la tragédie, aux yeux de Robbe-Grillet, elle aussi périmée.

\subsubsection{Éléments d'une anthologie moderne}
Les écrits théorique s'interrompent et s'intercalent un texte inédit ayant vocation de seuil. Aux récriminations contre une littérature périmées, aux recommendations techniques succèdent les cas pratiques~: nous sommes invités à lire la modernité à travers les lectures de Robbe-Grillet.

À la filiation prestigieuse (car constituée de classique relativement éloignés de son époque) se surimpose une filiation à la fois plus contemporaine, esquissant une voie pour la littérature. Si l'on ne pourrait affirmer avec certitude que Robbe-grillet défend l'idée d'un progrès en art, on observe du moins une gradation dans l'ordre de ces chroniques~: plutôt que d'être insérées dans leur ordre de publication ces critiques sont organisés de la plus éloignée à la plus proche de la modernité.

À travers ces critiques, Robbe-Grillet défend deux pendants de la modernité : des techniques ld'écriture, mais aussi une certaine attitude qu'il préconise à l'endroit des œuvres littéraire. Que devrait-on lire~? Comment devrait-on lire~? À ces questions ce court texte propose un début de réponse.


\subsubsection{Énigmes et transparences chez Raymond Roussel}
Publiée en décembre 1963 dans la revue \textit{Critique}, cette analyse du style de Raymond Roussel semble partir du constat que Raymond Roussel écrit «~mal~» c'est-à-dire qu'il ne répond pas aux habitudes de la réception. En effet, il ne s'agit pas d'une littérature du secret, ni même du dévoilement mais une littérature d'un imaginaire loufoque décrit avec la banalité de ce qui n'est que pour être. En cela ce cas pratique est à la fois une démonstration de la modernité selon Robbe-Grillet mais également de l'incapacité du discours de la critique contemporaine à rendre compte du fait littéraire, même relativement ancien.

\subsubsection{La conscience malade de Zeno}
Cette critique est parue initialement dans le numéro~19, juillet~1954 de la\textit{Nouvelle Revue Française}.

Présenté dans les premières lignes comme s'il s'agissait d'un roman classique au sens auquel l'entend Robbe-Grillet (une narration, un protagoniste, presque des épisodes, etc.), \textit{La Conscience de Zeno} se révèle un exemple de roman moderne.

Et si les thèses esthétiques de Robbe-Grillet n'y sont jamais détaillées ou même explicitement mentionnées, le lecteur attentif en décèle les indices. Le roman d'Italo Svevo traite du monde perçu depuis la conscience du protagoniste, c'est-à-dire de la conscience elle-même. L'œuvre se présente comme un journal non chronologique, donc plus proche d'un livre à part entière du fait de ce travail sur le temps, écrit à porpos de et par le protagoniste qui traite de son objet, lui-même sa conscience mais également du moyen pour la saisir, l'écriture elle-même. Robbe-Grillet invite à s'intéresser à la situation d'énonciation de l'ouvrage, devenu un roman du soupçon, posant la question~: «~Qui parle~?~»


\subsubsection{Joë Bousquet le rêveur}
Cette critique est la reprise d'une critique publiée pour la première fois dans le numéro~77, octobre~1954 de la revue \textit{Critique}.

À travers l'évocation de l'œuvre de Joë Bousquet, Robbe-Grillet illustre sa thèse sur la création littérature. La création littéraire n'est pas le fait d'une restitution, mais bien d'une création. Les objets ne sont pas symboles ou symptomes d'une profondeur inaccessible. Notons à ce propos que Robbe-Grillet cite un passage de Bousquet qui rapproche cette thèse de Roussel et du roman policier (page~108). Et s'il n'y a pas de profondeur, la signification existe mais toujours comme un «~sous-produit~» (p.~109) des choses elle-même qui n'y participe en aucun cas.

Bousquet est pour Robbe-Grillet l'exemple d'une littérature défaite de mysticisme, une illustration de la phénoménologie en littérature, défaite de métaphysique et surtout de tragédie~: si \textit{Le Meneur de lune} est pour Robbe-Grillet l'expression du hiatus entre le monde et l'homme, cela semble l'occasion d'une célébration et non d'une plainte.


\subsubsection{Samuel Beckett ou la présence sur la scène}
Constitué de deux critiques sur Beckett «~Samuel Beckett, Auteur dramatique~» et «~Samuel Becket ou la présence sur la scène~», respectivement parus dans les numéros~69 (février~1953) et~189 (février~1963) de \textit{Critique}, cet article confirme la gradation qu'on observe de critique en critique~: des précurseurs de la modernité on passe ici à ce que Robbe-Grillet considère être la modernité.

Lorsque Robbe-Grillet traite Beckett et \textit{a fortiori} à la parution du recueil, son théâtre est largement reconnu. Dès lors s'agit-il pour Robbe-Grillet de chercher la continuité d'une œuvre, sa direction pour les rattacher au projet du nouveau roman. Qui voit Robbe-Grillet~? Une lecture d'Heidegger qui coïncide avec ses thèses esthétiques, une contagion, un amenuisement progressif des choses, mais aussi un reflet de la condition humaine défait de tragédie et de métaphysique triste. Robbe-Grillet dépeint ici un théâtre neuf où le spectateur, loin de «~"pens[er]" ferme~» est stimulé tout au long de la représentation.


\subsubsection{Un roman qui s'invente lui-même}
Dans cet article, initialement paru dans \textit{Critique} en janvier~1954, Robbe-Grillet fait la critique de deux romans de Robert Pinget (auteur rattaché au nouveau roamn) \textit{Mahu ou le matériau} et \textit{Le Renard et la boussole} paru en 1952 et 1953. Robbe-Grillet s'y emploie à résumer l'intrigue autant que faire ce peut (le terme même d'«~intrigue~» semble ici inaproprié). Ce résumé vaut commentaire tant l'explicitation des détours de ces œuvres ne peut être que le récit d'une écriture.

Si, comme l'ensemble des critiques au sein de Pour un nouveau roman, le texte semble une digression dans l'économie de la démonstration, cette digression elle-même puisqu'elle est assemblée au recueil vaut d'emblée aux yeux du lecteur illustration des thèses.

Que nous démontre la lecture de Pinget que nous livre Robbe-Grillet~? Robbe-Grillet ne tire aucune conclusion dans cet article, cependant la relative inanité de l'exercice auquel il se livre ici, dénouer le fil de l'aventure, semble la meilleure incarnation possible des thèses exposées dans «~De quelques notions périmées~» : son analyse du personnage de Renard, le récit insaisissable se veulent les preuves que le discours habituel de la critique n'a pas prise sur les œuvres du nouveau roman% (et peut-être Robbe-Grillet non plus)
. Enfin il convient de souligner que l'article s'ouvre sur l'assertion initiale déplorant le fait que les œuvres de Pinget passe inaperçu. Outre le fait que cette affirmation semble corroborer les propos que Robbe-Grillet tient également dans «~Une voie pour le roman futur~».%, il n'est sans doute pas anodin que l'éditeur Robbe-Grillet publie à propos de Pinget publié chez Minuit.
    
    



\subsubsection{Nouveau roman, homme nouveau}
Bilan et relance de ce qui a été exposé dans le recueil, de ce qui a été écrit dans la presse «~Nouveau roman, homme nouveau~» initialement paru en~1961 dans le numéro~68 de la revue spécialisée \textit{Revue de Paris} s'affirme comme une lutte pas à pas, une subversion, «~un contre pied~» contre la \textit{doxa} fautive.

Robbe-Grillet rétablit (c'est-à-dire qu'il prescrit) la juste lecture du nouveau roman et de ses grands principes. Ce texte est sans doute le plus offensif, du moins le plus clairement sur une structure antagonistique vis-à-vis d'adversaires non désignés mais en lesquels on reconnaît \textit{Qu'est-ce que la littérature~?} de Sartre allant jusqu'à en reprendre la lettre «~Le seul engagement possible, pour l'écrivain, c'est la littérature~» (p.~152 chez Robbe-Grillet pour en proposer des contradictions frontales présentées comme des nuances ou des nuances présentées comme des contradictions frontales.


\subsubsection{Temps et description dans le récit d'aujourd'hui}
«~Temps et description dans le récit d'aujourdhui~» est en partie issue de la réécriture de «~Comment mesurer l'inventeur des mesures~?~» initialement publié dans \textit{L'Express} en juin 1963.

Contrairement à l'usage du recueil l'article s'ouvre sur une concession faite au critique~: il est difficile de penser la nouveauté. Aussi, Robbe-Grillet se propose t-il ici de leurs venir en aide en soumettant quelques théories liées à la technique de cette littérature nouvelle~: prenant le contre-pied de la critique qui rapproche le nouveau roman du cinéma, Robbe-Grillet s'emploie à démontrer l'intérêt des nouvelles techniques mises en œuvre dans la description et le traitement du temps. C'est encore une fois de l'histoire dont il est question, et du rapport qu'entretient le lecteur avec la narration.


\subsubsection{Du réalisme à la réalité}
            Continuant sur la lancé des articles plus théoriques que polémiques «~Du réalisme à la réalité~» est issue de quatre sources~:

\begin{itemize}

    \item «~La Littérature aujourd'hui~–~\textsc{vi}~», l'entretien de \textit{Tel Quel} publié en 1963

    \item «~Monsieur personne répond….Pour un ‘‘nouveau roman’’~», publié en~1963 dans \textit{Le Figaro Littéraire}

    \item «~Réalisme et révolution~», premier article publié en janvier~1955 de la série de chroniques de \textit{L'Express}.

    \item «~Pour un réalisme de la présence~», issu de la même série publié en janvier~1956.
\end{itemize}
De par sa place et son ton, cet article se présente comme une conclusion. Robbe-Grillet y traite du sens sinon de la direction de la littérature, partant d'une affirmation énoncé avec l'évidence d'un constat~: il n'y a de littérature que du réel et c'est précisèment ce souci de la réalité qui explique la succession des courants et des écoles littéraire et artistiques. Deux raisons sont brièvement esquissées~: les modes qui passent et le monde qui change. Mais la raison principale développée par Robbe-Grillet est la l'épuisement des formes qui se figent et devient convention. L'art véritable ne peut être que création et au nom d'une littérature plus réaliste Robbe-Grillet esquisse les contours d'un changement de paradigme~:l'invention n'est plus dissimulée mais bien exposée.

Prenons garde de ne pas mésinterpréter l'expression «~servir à quelque chose~». La valeur de la littérature n'est pas dans son inutilité, mais dans son incapacité à servir les idéologues(comprendre~: les engagés, les critiques, tous réactionnaires selon Robbe-Grillet) car le propre de la création réellement nouvelle est qu'elle subvertie toujours au nom d'une forme et d'un sens défaits de l'habitude et de la paresse.





 

%           le sens de la littérature, le progrès (?) l'histoire de litt est resaisi ici
    
    
%       part d'affirmation énoncé avec évidence d'un constat , il n'y a de littérature que du réel
%       moteur de l'histoire litt = la volonté de plus de réel, 
%       lorsque les genres littéraires/les écoles litt
%           deux raisons brèvement esquissées~: sucession des genres et  + le monde change
%           contre le réalisme en tant que genre, écoles devenu convention qui ne rend pas compte du réel mais d'une petite partie 
%           prend appui sur son inspiration son processus créatif pour critiquer l'idée de représenter le réel
%       changement de paradigme, l'invention ++ et le mensonge contre 
%       
%       critique via détour par Kafka des analyses littéraires qui cherche à ramener au connu des formes nouvelles
%       ==> le but de la littérature est donc de dégager des idées neuves grâce à une forme neuve imaginative
%       propos politique là dessus où il se peint à demi-mot en progressiste
%       s'ouvre sur une invitation explicite au lecteur et au romancier à rejoindre une ambition ++ pour la littérature
%       
%       
%       ne pas mésinterpréter « servir à qqch » qui pourrait être compris comme servir à qq1, les idéologues qui cherchent à produire une forme fermé pour un contenu préétabli
%       conclut sur notion esthétique forme/sens nouvelle
%       
    
    
%       redef de réalisme
    
    

\newpage

\section{Un style polémique}

Choisir d'écrire dans des quotidiens grand public ou même la presse spécialisée pour défendre son œuvre au nom de position esthétiques n'est pas un choix anodin \textit{a fortiori} lorsque tel choix se fait dans un environnement relativement hostile. \robbe{} fait le choix pleinement assumé d'entrer dans l'arène et d'y livrer une lutte sans merci pour imposer un nouveau langage critique, du moins acquérir une certaine reconnaissance. Le recueil constitué conserve le ton des articles et donne aux attaques et récréminations une relative cohérence.

\subsection{Des adversaires désignés ?}

    % utiliser la DB et la commenter

De par sa portée polémique, \punr{} se désigne des adversaires. Parmi lesquels on trouve les critiques généralement non nommés (à quelques exceptions près) et désignés par une expression «~les critiques~»~: ils sont les adversaires privilégiés et assimilés aux porteurs d'une \textit{doxa} à laquelle le texte s'oppose. Leurs propos réels ou supposés sont mis en valeur tout au long du texte et constituent l'un des points saillants que notre édition entend mettre en évidence (voir \ref{gloss}). Outre l'usage du discours indirect libre, de la citation (réelle ou fictive) on distingue l'usage de la périphrase pour désigner sans nommer, englober sans nuance les thèses adverses au sein de groupes dont les contours sont à la fois assez lâches pour se permettre une contestation facile et assez serrés pour permettre des charges précises.


\subsubsection{L'intérêt du sous-entendu}
Si l'on distingue quelques individualités nommés ou suggérés (voir \textit{infra}), les adversaires de Robbe-Grillet ne sont généralement pas nommés directement mais toujours désignés par des périphrases, telles page~29 «~nos bons critiques~» page 34 «~les gens sérieux~» ou des tournures impersonnelles page~33 «~On louera seulement le romancier~» ou page~16 «~Il ne semble guère raisonnable~». Nous pensons qu'il s'agit là d'un choix délibéré.

\begin{quote}
    D'autres critiques qui se montrent négatifs par rapport à son projet,
comme François Mauriac et André Rousseaux, par exemple, se voient
accorder une place d’honneur dans le recueil. Certains, comme Jean-
René Huguenin, n’y figurent qu’implicitement, fournissant toutefois à
Robbe-Grillet un cadre polémique dans lequel il peut avancer ses
thèses en réagissant à leurs remarques, et parfois même en les inventant,
quand elles ne sont pas exprimées en toutes lettres\footnote{\textit{Op. cit.}, p.~75}.
\end{quote}

Jean-René~Huguenin mais aussi Jean~Guéhenno sont deux auteurs qui prirent position publiquement contre le nouveau roman, tel le rapporte \galia\footnote{\textit{Op. cit.}, p.~137} le premier dans \textit{Le Figaro Littéraire}\footnote{Jean \textsc{Guéhenno}, «~Le roman de Monsieur Personne~»,\textit{ Le Figaro Littéraire}, 28 Nov.-4 déc. 1963}, le second dans la revue \textit{Arts}\footnote{Jean-René \textsc{Huguenin}, «~Le nouveau roman~: une mode qui passe~», \textit{Arts}, n°~836, 27~sept.-3~oct.~1961.}, pourquoi sont-ils épargnés alors que Mauriac et Rousseaux sont tous deux cités page~55~? 
%LA REF à Huguenin c'est p. 181
On aurait sans doute tort de penser qu'il s'agit là d'un procédé pour éviter de se faire de nouveaux ennemis, puisque ces deux auteurs se sont ouvertement positionnés contre le nouveau roman~; et que deux auteurs bien plus connus, eux, le sont. C'est que \punr, ne s'adresse pas à ces critiques et peut-être même pas au lecteur de la presse spécialisé mais à un lectorat plus large qui suit de loin, voire pas du tout, les polémiques littéraires. Le recueil bien que moins diffusé qu'un titre comme \textit{L'Express} s'adresse à un public plus large et possiblement à une époque éloignée de ces débats. 

En évitant de citer explicitement ses adversaires moins connus, \robbe{} leur témoigne un relatif mépris~: ils ne méritent pas d'être cités mais seulement d'être rejetés au sein d'une entité collective «~les critiques~» reflet de la \textit{doxa}, qui, par définition, ne sert qu'à être mise en échec par la démonstration du grand auteur (voir \ref{rhétoChemin}). Ainsi, les nuances de la pensée des adversaires, auxquels on peut dès lors prêter les propos que l'on souhaite (puisqu'il n'y a plus d'auteur identifié, nul n'est accusé à tort), sont gommées dans une masse informe et caricaturale qu'il est aisée pour \robbe{} de critiquer voire de moquer. %stratagème 1 de Schopenhauer

Mais surtout ce renvoi à une entité collective permet et c'est là l'enjeu principal d'un manifeste, de se constituer un public qui partage la même défiance à l'égard du champs littéraire hostile aux néo-romanciers, en s'attachant tous les déçus ou les lassés de la littérature contemporaine de laquelle \robbe{} entend se distinguer. Se constitue aux yeux du lecteur un bloc monolithique à la fois rétrograde «~Mais tous avouent, sans voir là rien d’anormal, que leurs préoccupations d’écrivains datent de plusieurs siècles~» (page~17)~; dominant, en témoigne l'adjonction des termes «~la plupart [des critiques, des auteurs]~» p.~17 p.~33 ou encore «~académiques~» p.~35 p.~86~; et leurs thèses objet de moquerie de l'auteur, tel le «~cœur romantique des choses~» page~23 (et plus généralement une grande partie des termes placés entre guillemet recensé dans notre index des notions adverses (voir~: \ref{gloss})). Par là, le lecteur est invité à épouser les thèses de \punr{} par dégoût pour les propos prêtés à ces «~critiques traditionnels~».



  %  Enfin notre étude entend dégager les raisons théoriques qui organiser la distribution des bons et des mauvais points par \robbe{} aux œuvres citées (telles \textit{La Nausée}, \textit{L'Étranger}, \textit{Le Parti pris des choses})  et aux critiques.



\subsubsection{Les critiques : tenants d'un ordre littéraire rétrograde}


%attaque ad nominem
%les critiques classiques
%les académiciens
Les adversaires principaux du recueil sont «~les critiques~», qui constituent davantage un groupe informe qu'un regroupement d'individus clairement définis. L'expression (au singulier ou au pluriel) revient à de nombreuses reprises, parfois accompagnée des qualificatifs «~traditionnels\footnote{\op29, p.~31 p.~162}~», «~académiques\footnote{\op36, p.~87, p.~174, p.~183}~» ou même «~bourgeoise\footnote{\op47, p.~174}~». Si ces termes, en particulier «~académiques~» permettent à tout lecteur d'imaginer les caractéristiques de cet adversaire, les propos prêtés à cette entité et aux (rares) individus cités qui lui sont rattachés permettent d'en préciser les contours. En effet, comme l'indique \galia, les propos de la critique et leurs attaques font l'objet d'une réécriture commentant la page~56 de \punr,
\begin{quote}
    Que des écrivains aussi différents que François Mauriac et André Rousseaux, par exemple, s’accordassent à dénoncer dans la description exclusive des «~surfaces~» une mutilation gratuite, un aveuglement de jeune révolté, une sorte de désespoir stérile qui conduisait à la destruction de l’art, cela paraissait néanmoins dans l’ordre.
\end{quote}
\galia{} observe~:
\begin{quote}
    La paraphrase est plutôt fictive dans la mesure où elle ne reflète pas la véritable critique de Rousseaux et de Mauriac (les adjectifs mentionnés ci-dessus ne figurent pas dans leurs articles respectifs). C’est-à-dire qu’il met dans la bouche de ses critiques des termes hyperboliques\footnote{\fullgalia116}
\end{quote}
Qu'est-ce qui constitue ce groupe~? Tout ce qui s'oppose à la modernité du roman~:
\begin{itemize}
    \item «~un vocabulaire~» qui trahit un système\footnote{\op29} constitué de «~notions périmées~» et qualifié page~30 de «~mots magiques~».
    \item un attachement immodéré et coupable au personnage\footnote{\op31}, à l'histoire\footnote{\op34}, à l'opposition entre le fond et à la forme\footnote{\op47}, le premier prenant le pas sur la seconde sous la forme de l'engagement\footnote{\op39}.
    \item une aversion pour le «~formalisme\footnote{\op49}~».
    \item une méconception sur l'humanisme qui le confond avec la métaphysique
    \begin{quote}
        Condamner, au nom de l’humain, le roman qui met en scène un tel homme, c’est donc adopter le point de vue \textit{humaniste}, selon lequel il ne suffit pas de montrer l’homme là où il est~: il faut encore proclamer que l’homme est partout. Sous prétexte que l’homme ne peut prendre du monde qu’une connaissance subjective, l’humanisme décide de choisir l’homme comme justification de tout. Véritable pont d’âme jeté entre l’homme et les choses, le regard de l’humanisme est avant tout le gage d’une solidarité\footnote{\op59}.
    \end{quote} 
    %\item la souscription sans réserve, du moins une nostalgie, pour «~les idéologies trompeuses et les mythes~» ou les «~"valeurs" terrestres de la société bourgeoise les recours magiques, religieux ou philosophiques à tout « au-delà » spirituel de notre monde visible~» qui pourtant «~ont disparu\footnote{\op44}~» %CONCERNE LE RÉALISME SOCIALISTE ÇA VA PAS
\end{itemize}

En somme, les critiques sont naïfs page~31 \robbe{} ironise «~Un personnage, tout le monde sait ce que le mot signifie. Ce n’est pas un il quelconque, anonyme et translucide, simple sujet de l’action exprimée par le verbe~» avant de décrire le personnage balzacien. Observons que la non-définition proposée du personnage en plus d'être un résumé du personnage type dans le Nouveau~Roman (bel et bien effecctive rendant donc l'argumentaire supposé des critiques caduque), pourrait aussi bien être la description du personnage en tant qu'instance textuel, en quelque sorte plus sérieuse que la description du personnage balzacien qui semble dès lors une peinture naïve de la théorie littéraire. %Il va sans dire qu'ainsi Robbe-Grillet se donne une image de spécialiste plus compétent~: puisque les critiques sont naïfs et si aisément ridiculisés (cf \textit{infra} le cas d'Henri~Clouard) et qu'\robbe{} est là pour le souligner, il n'en paraît que plus sérieux.
%EXEMPLE

Les adversaires de Robbe-Grillet sont si souvent invoqués comme source d'un propos auquel il s'oppose que toute pensée contre laquelle se dresse \robbe{} semble constituer, sauf indication contraire, l'une des thèses de «~la critique~». Ainsi si la tournure impersonnelle domine dans quelque passage de «~Nouveau roman, homme nouveau~», il ne fait aucun doute que cet impersonnel rétrograde est synonyme de «~les critiques~»~:
\begin{quote}
    L'erreur est de croire que le «~vrai roman~» s’est figé une fois pour toutes, à l’époque balzacienne, en des règles strictes et définitive s[...]. La construction de nos livres n’est d’ailleurs déroutante que si l’on s’acharne à y rechercher la trace d’éléments qui ont en fait disparu [...]\footnote{\op146}

    Comme il n’y avait pas, dans nos livres, de «~personnages~» au sens traditionnel du mot, on en a conclu, un peu hâtivement, qu’on n’y rencontrait pas d’hommes du tout. C’était bien mal les lire.\footnote{\op147}
\end{quote}

On peut tout de même parmi «~les critiques~» pris comme un bloc faire des recoupements et identifier quelques auteurs parfois cités.
On identifie Mauriac et Rousseaux cités page~56~; mais aussi Émile~Henriot. Après l'attribution du prix des critiques au roman \textit{Le Voyeur}, Émile Henriot académicien, écrit dans \textit{Le Monde}~:
\begin{quote}
    Le voyeur de M.~Robbe-Grillet n'alléchera personne sans le décevoir~: c'est seulement un type indécis, incertain, peut-être secret ou qui ne comprend rien non plus à son affaire\footnote{\textsc{Henriot} Émile, «~Le prix des critiques "Le Voyeur", d'Alain Robbe-Grillet~», Paris, \textit{Le Monde}, 15~juin~1955, [En ligne~: \href{https://www.lemonde.fr/archives/article/1955/06/15/le-prix-des-critiques-le-voyeur-d-alain-robbe-grillet_1958094_1819218.html}{https://www.lemonde.fr/archives/article/1955/06/15/le-prix-des-critiques-le-voyeur-d-alain-robbe-grillet\_1958094\_1819218.html}]}
\end{quote}
Et s'il semble bien que les reproches qu'Émile~Henriot a fait à \robbe{} ont pu nourrir l'œuvre, il n'est jamais cité. On peut cependant inférer que des reproches tels

\begin{quote}
    En, voici le procédé de M.~\robbe~: il va et vient dans son récit, avance, recule, change de lieu, ne tient aucun compte du temps. On est à telle heure et tant de minutes, et trois pages ou dix lignes plus loin on est revenu en arrière, et tout est mêlé comme dans l'esprit inquiet de Mathias, l'énigmatique voyageur marchand de montres-bracelet, assassin possible\footnote{\op\textsc{Henriot} Émile, «~Le prix des critiques "Le Voyeur", d'Alain Robbe-Grillet~», Paris, \textit{Le Monde}, 15~juin~1955, [En ligne~: \href{https://www.lemonde.fr/archives/article/1955/06/15/le-prix-des-critiques-le-voyeur-d-alain-robbe-grillet_1958094_1819218.html}{https://www.lemonde.fr/archives/article/1955/06/15/le-prix-des-critiques-le-voyeur-d-alain-robbe-grillet\_1958094\_1819218.html}]}.
\end{quote}
inspirent peut-être certains passages~:
\begin{quote}
    Tandis que, dans le récit moderne, on dirait que le temps se trouve coupé de sa temporalité. Il ne coule plus\footnote{\op168}.
\end{quote}
Plus sûrement, Émile~Henriot n'est pas tout à fait incapable de lire et la description qu'il fait de \textit{Le Voyeur} pour le dénigrer est la même que Robbe-Grillet (ou n'importe quel lecteur compétent) dresserait pour en faire l'éloge. Dès lors il semble contre productif de citer Émile~Henriot explicitement, s'il faut lui donner en partie raison, la peinture des critiques académiques en imbéciles perd de sa force. Cependant la récurrence des attaques contre l'Académie~Française pourrait être interprété comme un reste de présence d'Émile~Henriot au sein du recueil~: on peut supposer qu'une aversion peut-être nourrie par des positions de fait antithétique n'a pu qu'être accentuée par la prise de position d'Émile~Henriot, au moment où \robbe{} accédait à un semblant de reconnaissance. Notamment ce passage qui suit l'explication du traitement du temps dans \textit{La Jalousie}~: «~Et cela non pas dans le but stupide de dérouter l'Académie\footnote{\op167}~».

D'après \galia{} une même mention implicite fortement suggérée peut être inférée à l'endroit d'Alain~Bosquet qui «~fait de Robbe-Grillet un cas de figure négatif de l’écriture d’avant-garde\footnote{\fullgalia105}~» dans un article paru dans la revue \textit{Preuves\footnote{\textsc{Bosquet} Alain, «~Roman d’avant-garde et antiroman~», \textit{Preuves}, n°~79, septembre~1957, p.~79-86}}. C'est ce critique qu'il faut reconnaître au début du troisième article «~Sur quelques notions périmées~» page~30~:
\begin{quote}
    Dès qu'un écrivain renonce aux formules usées pour tenter de forger sa propre écriture, il se voit aussitôt coller l'étiquette~: "avant-garde". [...] le lecteur, averti par un clin d'œil, pense aussitôtà quelques jeunes gens hirsutes qui s'en vont, le sourire en coin, placer des pétards sous les fauteuils de l'Académie, dans le seul but de faire du bruit ou d'épater les bourgeois.
\end{quote}

Sont explicitement cités trois autres critiques~: Henri~Clouard un académicien cité (une citation extraite d'une publication qui n'a pas pu être identifiée) pour être ridiculiser dans la suite directe de l'extrait précédemment cité, François~Mauriac et André~Rousseaux dont les articles parus dans \textit{Le Figaro Littéraire}\footnote{\textsc{Rousseaux}~André, «~Les Surfaces d’Alain~Robbe-Grillet~», \textit{Le Figaro littéraire}, 13~avril~1957, p.~2 et \textsc{Mauriac}~François, «~Technique du cageot~», \textit{Le Figaro littéraire}, 28~juillet~1956, p.~1-3.} sont, d'après \galia{} exagérés pour être mieux réfuter\footnote{\fullgalia116}. 
%p. 120-121 sur la constitution théorique de « NAture humanisme etc contre MAuriac
%Henri Clouard est un fasiste Action Française etc et son pote c'est Émile Henriot qui en fait un éloge dans le monde en 49.... comme quoi peu importe de quel côté on prend les choses il vaut mieux être ARG que l'un de ces toquards. (et l'uique source de wiki 'est le travail d'un litt/philosophe qui écrit sur les figures chrétiennes, maurras et contre le libéralisme... lol)


\subsubsection{Les « réalistes socialistes »}

%%%PARLER DES ENGAGEÉS ETC
Écrits dans les années~1950-60, \punr{} se moque des bourgeois et accuse ses adversaires de «~réalisme socialisme~», soit d'appartenir à un art d'état communiste censé faire advenir mais surtout accompagner la dictature du prolétariat. L'accusation est quelque peu hyperbolique et, ici encore, le fait de ne pas donner de noms permet de brasser large sans risquer d'être à son tour attaqué. Cependant Robbe-Grillet semble prêt à concéder au mouvement quelque intérêt~:
\begin{quote}
    Tout cependant n’était pas a priori condamnable dans cette théorie soviétique dite du «~réalisme socialiste~». En littérature, par exemple, ne s’agissait-il pas aussi de réagir contre une accumulation de fausse philosophie qui avait fini par tout envahir, de la poésie au roman~? S’opposant aux allégories métaphysiques, luttant aussi bien contre les arrière-mondes abstraits que celles-ci supposent que contre le délire verbal sans objet ou le vague sentimentalisme des passions, le réalisme socialiste pouvait avoir une saine influence.

    Ici n’ont plus cours les idéologies trompeuses et les mythes. La littérature expose simplement la situation de l’homme et de l’univers avec lequel il est aux prises. En même temps que les «~valeurs~» terrestres de la société bourgeoise ont disparu les recours magiques, religieux ou philosophiques à tout «~au-delà~» spirituel de notre monde visible. Les thèmes, devenus à la mode, du désespoir et de l’absurde sont dénoncés comme des alibis trop faciles. Ainsi Ilya~Ehrenburg ne craignait-il pas d’écrire au lendemain de la guerre~: «~L’angoisse est un vice bourgeois. Nous, nous reconstruisons\footnote{\op44}.~»
\end{quote}
La promesse destitution de la profondeur et des vieux mythes donc des postulats philosophique proches de ceux Robbe-Grillet au nom d'une esthétique neuve car défaite de vices rétrogrades a de quoi séduire \robbe{}~; cependant le passage au conditionnel dans le paragraphe suivant consacre le passage à une subtile ironie et anticipe le rejet en bloc page~45~: «~Quel progrès y a-t-il si, pour échapper au dédoublement des apparences et des essences, on tombe dans un manichéisme du bien et du mal ?~».

S'il ne se penhe que sur \punr{}, le lecteur ne sait pas tellement ce qui constitue la doctrine des réalistes socialistes, \robbe{} en appelle à des représentations partagées au nom d'une connivence développée sur les attaques contre ce groupe et les critiques bourgeois. Cependant les reproches que le texte fait aux réalistes socialistes permet au lecteur de concevoir, non seulement les torts des réalistes socialistes (portée explicite du passage) mais aussi les thèses qu'ils défendent (portée implicite)~:
\begin{quote}
Or, du point de vue de la littérature, les vérités économiques, les théories marxistes sur la plus-value et l’usurpation sont aussi des arrière-mondes. Si les romans progressistes ne doivent avoir de réalité que par rapport à ces explications fonctionnelles du monde visible, préparées d’avance, éprouvées, reconnues, on voit mal quel pourrait être leur pouvoir de découverte ou d’invention~; et, surtout, ce ne serait une fois de plus qu’une nouvelle façon de refuser au monde sa qualité la plus sûre~: le simple fait qu’il est là\footnote{\op45-46}.
\end{quote}
Mais il y a plus, au fil du texte se révèle une homologie structurale entre les engagés et les réalistes socialistes. À la sous-section %suivante
«~La forme et le contenu~», le terme «~engagé~» disparaît au profit de «~partisans du réalisme socialiste~» (page~47), «~serviteurs de Jdanov~» (page~48), la fusion des deux groupes est définitivement consommée page~50 «~les romans "engagés" qui se prétendent révolutionnaires~», par un zeugme, Robbe-Grillet reproche une littérature rétrograde (de par sa pauvreté formelle) aux engagés tout en faisant signe vers une littérature censée soutenir la révolution. Par ailleurs, l'argument ne manque pas d'intérêt~: 
\begin{quote}
    Il est dès lors normal que l’accusation de «~formalisme~» soit l’une des plus graves dans la bouche de nos censeurs des deux bords. Cette fois encore, malgré qu’ils en aient, c’est une décision systématique sur le roman que le mot révèle~; et, cette fois encore, sous son air naturel, le système cache les pires abstractions~–~pour ne pas dire les pires absurdités. On peut en outre y déceler un certain mépris de la littérature, implicite, mais flagrant, qui étonne autant venant de ses défenseurs officiels~–~les conservateurs de l’art et de la tradition~– que de ceux qui ont fait de la culture des masses leur cheval de bataille favori.

    Qu’entendent-ils au juste par formalisme~? La chose est claire~: ce serait un souci trop marqué de la forme –~et, dans le cas précis, de la technique romanesque~– aux dépens de l’histoire et de sa signification.
\end{quote}
puisque les «~partisan du réalisme socialiste~» et les «~critiques bourgeois les plus endurcis~» considèrent la littérature par les mêmes outils (distinguant forme et fond), ce sont les mêmes et il n'est plus utile de les distinguer. Et, une fois les critiques bourgeois et les réalistes socialistes déclarés similaire, on voit mal comment les engagés seraient épargnés, en témoigne la page~40~:
\begin{quote}
    Le roman à thèse est même rapidement devenu un genre honni entre tous... On l’a pourtant vu, il y a quelques années, renaître à gauche sous de nouveaux habits~: «~l’engagement~»~; et c’est aussi, à l’Est et avec des couleurs plus naïves, le «~réalisme socialiste~».
\end{quote}

Écrivant au fil de la pensée, du moins le laissant entendre, l'écriture de Robbe-Grillet s'embarrasse de peu de concessions, par exemple page~46, le résumé aux airs de concession qu'il fait de l'engagement Sartrien évoque le réalisme socialisme et toute concession faite à l'un servira à dénigrer l'autre~:
%Si cette affirmation semble à nuancer au vu de la page~46, une concession faite à Sartre~:
\begin{quote}
    Que reste-t-il alors de l’engagement~? Sartre, qui avait vu le danger de cette littérature moralisatrice, avait prêché pour une littérature \textit{morale}, qui prétendait seulement éveiller des consciences politiques en posant les problèmes de notre société, mais qui aurait échappé à l’esprit de propagande en rétablissant le lecteur dans sa liberté. L’expérience a montré que c’était là encore une utopie~: dès qu’apparaît le souci de signifier quelque chose (quelque chose d’extérieur à l’art) la littérature commence à reculer, à disparaître.
\end{quote}



Les adversaires nommés ou désignés de manière si précise que l'implicite s'expose en procédé rhétorique, semblent constituer des \textit{exempla} de groupes adversaires. Ainsi Sartre dont l'éloge nuancé pour \textit{La Nausée} semble \textit{exemplum} du groupe «~socialistes révolutionnnaires~». En effet Robbe-Grillet semble ne pas faire de distinction entre «~les engagés~» et «~les réalistes socialistes~», ainsi lit-on page~40 à propos de la notion périmée d'engagement~:
\begin{quote}
    On l’a pourtant vu, il y a quelques années, renaître à gauche sous de nouveaux habits~: «~l’engagement~»~; et c’est aussi, à l’Est et avec des couleurs plus naïves, le «~réalisme socialiste~».
\end{quote}
Exemple type des références à l'engagement dans \punr{} renvoyant de manière implicite à l'ouvrage de Sartre \textit{Qu'est-ce que la littérature}\footnote{\textsc{Sartre} Jean-Paul, \textit{Qu'est-ce que la littérature}, Paris, Gallimard, coll. «~folio essais~», 2008 [1948]} paru en 1948 (ou à la série d'articles du même nom publié de février à juillet 1947 dans \textit{Les Temps modernes} (voir~: \ref{vsSartre}), le raccourci nous paraît manifeste. Sartre pour se défendre de ces détracteurs qui lui reprochent «~d'engager~» la littérature, prend appui sur le sens commun, «~N'a-t-on pas coutume à tous les jeunes gens qui se proposent d'écrire cette question de principe~: "Avez-vous quelque chose à dire~?"\footnote{\textsc{Sartre} Jean-Paul, \textit{Qu'est-ce que la littérature}, Paris, Gallimard, coll. «~folio essais~», 2008 [1948], p.~27}~» pour conclure «~Il faut bien, de leur [les critiques] aveu même, que l'écrivain parle de quelque chose\footnote{\textsc{Sartre} Jean-Paul, \textit{Qu'est-ce que la littérature}, Paris, Gallimard, coll. «~folio essais~», 2008 [1948], p.~32}~». Sartre affirme la prééminence du fond sur la forme pour justifier la nécessité pour l'écrivain de s'engager, c'est précisément en jouant sur cet argument non explicité que Robbe-Grillet reproche aux deux groupes page~47~:
\begin{quote}
    Une chose devrait troubler les partisans du réalisme socialiste, c'est la parfaite ressemblance de leurs arguments, de leur vocabulaire, de leurs valeurs, avec ceux des critiques bourgeois les plus endurcis. Par exemple lorsqu'il s'agit de séparer la "forme" d'un roman de son "contenu", c'est-à-dire d'opposer l'\textit{écriture} (choix des mots et leur ordonnance, emploi des temps grammaticaux et des personnes, structure du récit, etc.) à l'anecdote qu'elle sert à rapporter (événements, actions des personnages, motivations de celles-ci, morale qui s'en dégage).
\end{quote}
\robbe{} ici consacre le fait que les réalistes socialistes sont bourgeois et démontre que son opposition principale aux uns et aux autres est le souci de la recherche formelle.





\subsection{Une technique argumentative : le cheminement}
\label{rhétoChemin}

\punr{} étant la somme d'articles publiés dans la presse, la voix de l'essayiste est fortement incarnée et l'argumentaire tente de s'approcher d'un cheminement progressif où la pensée se fait jour à elle-même. Si deux articles arbore une structure claire et un programme définie («~Sur quelques notions périmées~» et «~Nouveau roman, homme nouveau~»), tous sont construits selon un patron général identique~: l'exposition de la thèse adverse puis sa réfutation, répété autant de fois que nécessaire au sein d'un article, si nécessaire, \textit{via} le détour par des concessions à leur tour réfutées. Est mimée ainsi une conversation avec la critique, dans laquelle le lecteur identifie bien-sûr les étapes d'une pensée, une délibération interne.

Ces étapes sont mises en valeur dans notre édition numérique et permet au lecteur d'identifier promptement les moments de la délibération.

%SI TEMPS %SI TEMPS%SI TEMPS %SI TEMPS%SI TEMPS %SI TEMPS%SI TEMPS %SI TEMPS%SI TEMPS %SI TEMPS%SI TEMPS %SI TEMPS%SI TEMPS %SI TEMPS%SI TEMPS %SI TEMPS%SI TEMPS %SI TEMPS%SI TEMPS %SI TEMPS%SI TEMPS %SI TEMPS%SI TEMPS %SI TEMPS%SI TEMPS %SI TEMPS%SI TEMPS %SI TEMPS%SI TEMPS %SI TEMPS%SI TEMPS %SI TEMPS%SI TEMPS %SI TEMPS%SI TEMPS %SI TEMPS%SI TEMPS %SI TEMPS%SI TEMPS %SI TEMPS%SI TEMPS %SI TEMPS%SI TEMPS %SI TEMPS%SI TEMPS %SI TEMPS%SI TEMPS %SI TEMPS%SI TEMPS %SI TEMPS%SI TEMPS %SI TEMPS%SI TEMPS %SI TEMPS%SI TEMPS %SI TEMPS%SI TEMPS %SI TEMPS%SI TEMPS %SI TEMPS%SI TEMPS %SI TEMPS%SI TEMPS %SI TEMPS%SI TEMPS %SI TEMPS%SI TEMPS %SI TEMPS%SI TEMPS %SI TEMPS%SI TEMPS %SI TEMPS%SI TEMPS %SI TEMPS%SI TEMPS %SI TEMPS%SI TEMPS %SI TEMPS%SI TEMPS %SI TEMPS%SI TEMPS %SI TEMPS%SI TEMPS %SI TEMPS%SI TEMPS %SI TEMPS%SI TEMPS %SI TEMPS%SI TEMPS %SI TEMPS%SI TEMPS %SI TEMPS%SI TEMPS %SI TEMPS%SI TEMPS %SI TEMPS%SI TEMPS %SI TEMPS%SI TEMPS %SI TEMPS%SI TEMPS %SI TEMPS%SI TEMPS %SI TEMPS%SI TEMPS %SI TEMPS%SI TEMPS %SI TEMPS%SI TEMPS %SI TEMPS%SI TEMPS %SI TEMPS%SI TEMPS %SI TEMPS%SI TEMPS %SI TEMPS%SI TEMPS %SI TEMPS%SI TEMPS %SI TEMPS%SI TEMPS %SI TEMPS%SI TEMPS %SI TEMPS%SI TEMPS %SI TEMPS%SI TEMPS %SI TEMPS
%exemple en gros de deux chapitre qui détaille le chemin et fai lien avec édition numérique

Outre le principe, somme toute classique en rhétorique, d'exposer la thèse adverse que l'on s'emploie à défaire, il convient de noter que les thèses adverses sont le véritable moteur de l'argumentation. Citant un entretien de Robbe-Grillet issu du numéro~764 de \textit{Les Lettres françaises}, \galia{} que les critiques nourrissent la pensée de l'écrivain \robbe~: «~il peut y avoir un dialogue entre romanciers et critiques. Il est même quelquefois très profitable\footnote{Anne \textsc{Villelaur}, «~Le nouveau roman est en train de réfléchir sur lui-même~», \textit{Les Lettres françaises}, n°~764, 12-18.03.1959, p.~4}~». Le débat est mis en scène dans l'écriture et en retour la met en branle, d'une part par le phénomène d'exposition/réfutation mais aussi par les multiples commentaires digressifs qui prennent appui sur la pensée adverse~: 
\begin{quote}
À moins d’estimer que le monde est désormais entièrement découvert (et, dans ce cas, le plus sage serait de s’arrêter tout à faire d’écrire), on ne peut que tenter d’aller plus loin\footnote{\op 172}.    
\end{quote}
Chaque expression prêtée à l'adversaire fait l'objet d'une réfutation montrée comme immédiate et donc spontanée. De par son statut digressif, l'usage des parenthèses semblent sous-tendre une incise dans une pensée en mouvement, l'inspiration soudaine dans le souffle d'une théorie qui s'achemine à force de réfuter les thèses adverses. Ainsi, l'écriture de Robbe-Grillet est-elle dans \punr{} faites de rebonds, sur et à propos de critiques adressés à ces écrits, du moins de positions qu'il ne partage pas. Suite immédiate du passage précédemment cité, l'essayiste avance pas à pas sur les traces de ces adversaires~:
\begin{quote}
    Il ne s’agit pas de «~faire mieux~», mais de [...]
    
    À quoi cela sert-il, dira-t-on, [...]
    
    La critique académique, à l’Ouest comme dans les pays communistes, emploie le mot « réalisme » comme si[...]\footnote{\op 172-173}
\end{quote}

Ainsi progresse l'exposition des théorie de Robbe-Grillet, en réaction à des critiques qu'il se fait lui-même, comme si elle venait rectifier les erreurs ou les biais idéologiques des critiques.

%Nous tirons, à dessein, cette citation du dernier chapitre du recueil «~Du réalisme à la réalité~», dans lequel la technique de l'exposition/réfutation ne structure pas l'article. Souhaité conclusif, la réfutation est bien présente mais plus diffuse c'est que pour une fois les thèses adverses ne font pas le cœur de l'argumentation. Au contraire, ici, \robbe{} déroule sa pensée. Or, si les thèses adverses ne sont pas exposées avec la même ampleur que dans les premiers articles du recueil, elles sont bien présentes et utilisées pour relancer l'argumentation qui semble ne pouvoir se faire sans elles~: 

%Les articles ou segments argumentatifs se conclut ensuite sur un bilan qui ouvre et justifie la réfutation en semblant en découler.


\subsubsection{L'histoire d'une dispute}
\label{histoireDispute}
Sous l'impulsion des détours et des rebonds la pensée se fait presque narration. L'emploi du discours indirect libre et la capacité de Robbe-Grillet à laisser une place aux discours adverses dont le lecteur perçoit aisément qu'ils sont (re)constitués pour les besoins de la mise en scène du débat~: 
\begin{quote}
    Il ne semble guère raisonnable, à première vue, de penser qu’une littérature entièrement \textit{nouvelle} soit un jour – maintenant, par exemple – possible. Les nombreuses tentatives, qui se sont succédé depuis plus de trente ans, pour faire sortir le récit de ses ornières n’ont abouti, au mieux, qu’à des œuvres isolées\footnote{\textit{Op. cit.}, p.~16}.
\end{quote}transforme par moment l'essayiste en un narrateur omniscient qui se plaît à commenter avec force ironie et force affectivité.
\begin{quote}
    Non seulement le livre déplut et fut considéré comme une sorte d’attentat saugrenu contre les belles-lettres, mais on démontra de surcroît comment il était normal qu’il fût à ce point exécrable, puisqu’il s’avouait le produit de la préméditation~: son auteur –~ô scandale~!~– se permettait d’avoir des opinions sur son propre métier\footnote{\textit{Op. cit.}, p.~10}.
\end{quote}
L'emploi du passé simple et les antagonismes fort qui sous-tendent toute l'œuvre font du recueil d'essai, du moins de ces premiers chapitres qui concentrent ces effets utiles, sans doute aux yeux d'\robbe{} à l'exposition des termes du débats, une histoire. Le lecteur est ainsi invité à assister à l'émergence du Nouveau Roman devenu par moment l'objet d'une joute épique entre le narrateur et les antagonistes qu'il s'est constitué.

De même page~39, le récit, à mi-chemin entre le conte et la rhétorique, devient l'outil d'un argument qui vaut conclusion~:
\begin{quote}
Mais j’imagine sans mal que dans quelques dizaines d’années~–~plus tôt peut-être~–~lorsque cette écriture, assimilée, en voie de devenir académique, passera inaperçue à son tour, et qu’il s’agira bien entendu pour les jeunes romanciers de faire autre chose, la critique d’alors, trouvant une fois de plus qu’il ne se passe rien dans leurs livres, leur reprochera leur manque d’imagination et leur montrera nos romans en exemple~: «~Voyez, diront-ils, comme, dans les années cinquante, on savait inventer des histoires~!~»
\end{quote}
autre saynète, qui n'est pas sans rappeler \textit{L'Ére du soupçon}~:
\begin{quote}
Puisque raconter pour distraire est futile et que raconter pour faire croire est devenu suspect, le romancier pense apercevoir une autre voie~: raconter pour enseigner. Las de s’entendre déclarer avec condescendance par les gens assis~: «~Je ne lis plus de romans, j’ai passé l’âge, c’est bon pour les femmes (qui n’ont rien à faire), je préfère la réalité...~» et autres niaiseries,  romancier va se rabattre sur la littérature didactique.   
\end{quote}
Lecteurs potentiels, critique actuel ou à venir et romancier deviennent ainsi les personnages d'un argumentaire, permettant de tenir de prêter aux uns et aux autres des attitudes comiques, outrancières, grotesques afin d'en montrer l'inanité, tout en ayant l'air de se placer en position empathique vis-à-vis de ses adversaires. Notons que le caractère très général de ces anecdotes demandent au lecteur d'investir de son vécu ces figures et est ainsi à même de produire une certaine connivence entre ce dernier et \robbe. 


\subsubsection{Robbe-Grillet, défenseur du bon sens ?}
\label{bonsens}

De par la structure rhétorique (voir \textit{supra} \ref{rhétoChemin}) qui laisse entendre que les positionss de Robbe-Grillet émerge d'une délibération, celui-ci se positionne clairement du côté du bon sens. 

Or, ce concept de «~bon sens~» mérite d'être interrogé, du moins défini. Si tous s'en réclament, chacun y projette un rapport particulier au savoir. Dans le cas d'\robbe{} le bon sens n'est pas la \textit{doxa}, un avis partagé immédiatement par le plus grand nombre mais bien le résultat d'un dénuement, d'un dévoilement, permettant d'accéder à la vérité (non identique au réel) qui n'étant pas immédiatement donnée doit faire l'objet d'un travail (de la pensée ou de la forme) pour défaire l'habitude qui prend tantôt l'apparence de clichés tantôt celle de la tradition. Cette tradition n'est pas seulement une position adverse, un autre moyen d'accéder à la réalité mais un voile tissé d'illusion qui recouvre le réel. En cela, le bon sens s'il s'obtient en dépit d'un premier mouvement ne peut être confondu avec une «~profondeur~» ou un secret enfoui dans les choses, c'est l'évidence des faits dissimulée sous des positions de principes fautives. 

De là, la nécessité de rappeler que \robbe{} ne se considère pas comme un théoricien
\begin{quote}
    Je ne suis pas un théoricien du roman\footnote{\op7}.


    Ainsi, loin d’édicter des règles, des théories, des lois, ni pour les autres ni pour nous-mêmes, c’est au contraire dans la lutte contre des lois trop rigides que nous nous sommes rencontrés\footnote{\op145}.

    Mais nous, au contraire, qu’on accuse d’être des théoriciens, nous ne savons pas ce que doit être un roman, un vrai roman ; nous savons seulement que le roman d’aujourd’hui sera ce que nous le ferons, aujourd’hui, et que nous n’avons pas à cultiver la ressemblance avec ce qu’il était hier, mais à nous avancer plus loin\footnote{\op146}.
\end{quote}
à l'inverse de ces détracteurs dont les positions sont artificielles, construites en dépit d'un état de fait palpable
\begin{quote}
    Le récit, tel que le conçoivent nos critiques académiques – et bien des lecteurs à leur suite – représente un ordre. Cet ordre, que l’on peut en effet qualifier de naturel, est lié à tout un système, rationaliste et organisateur, dont l’épanouissement correspond à la prise du pouvoir par la classe bourgeoise. En cette première moitié du \textsc{xix}\textsuperscript{e} siècle, qui vit l’apogée – avec \textit{la Comédie humaine} – d’une forme narrative dont on comprend qu’elle demeure pour beaucoup comme un paradis perdu du roman, quelques certitudes importantes avaient cours : la confiance en particulier dans une logique des choses juste et universelle.
    [...]
    Pourtant, là encore, il suffit de lire les grands romans du début de notre siècle pour constater que, si la désagrégation de l’intrigue n’a fait que se préciser au cours des dernières années, elle avait déjà cessé depuis longtemps de constituer l’armature du récit.
    \footnote{\op36-37}.
\end{quote}
Les thèses des adversaires de Robbe-Grillet semblent devoir être toujours fausses puisqu'elles procèdent par la suimposition au réel de grilles de lecture predéterminées, de clichets~:
\begin{quote}
    La construction de nos livres n’est d’ailleurs déroutante que si l’on s’acharne à y rechercher la trace d’éléments qui ont en fait disparu depuis vingt, trente, ou quarante années, de tout roman vivant, ou se sont du moins singulièrement effrités : les caractères, la chronologie, les études sociologiques, etc\footnote{\op146}.

    Comme il n’y avait pas, dans nos livres, de « personnages » au sens traditionnel du mot, on en a conclu, un peu hâtivement, qu’on n’y rencontrait pas d’hommes du tout. C’était bien mal les lire\footnote{\op147}.
\end{quote}
Au contraire les thèses de Robbe-Grillet sont exposée comme une acceptation de ce principe tel  fréquente introduction par des modalisateurs parfois adjonts à des tournures impersonnelles le souligne~:

\begin{quote}
    Pourtant, là encore, il suffit de lire\footnote{\op37}


    Malheureusement, elle ne convainc plus personne\footnote{\op39}


    Hélas, dès que l’on passe à la pratique, les choses se gâtent\footnote{\op41}.

    Regardons maintenant le résultat. Que nous offre le réalisme socialiste ?\footnote{\op45}

    Qu’il n’y ait qu’un parallélisme assez lâche entre les trois romans que j’ai publiés à ce jour et mes vues théoriques sur un possible roman futur, c’est l’évidence même\footnote{\op56}.

    Malheureusement, parmi les critiques qu’on lui prodiguait, et aussi, souvent, parmi les éloges, il y avait tant de simplifications extrêmes, tant d’erreurs, tant de malentendus, qu’une sorte de mythe monstrueux a fini par se constituer\footnote{\op143}
\end{quote}
Ces débuts ou relances de réfutation au ton cinglant illustre un fait~: la conclusion est toujours déjà tirée avant d'être exprimée.

Notons que les moyens de la justification de la position théorique (l'appel au bon sens en tant que dévoilement) coïncide avec la théorie littéraire d'\robbe{} où le réel est ce qui résiste et la signification toujours artificielle (voir \textit{infra} \ref{3phéno}).

De même, l'article «~Nouveau roman, homme nouveau~» prenant le contrepied exprime clairement la méthode (prendre le contre-pied), si le procédé repose sur un argument facile il n'en demeure pas moins juste~: les adversaires ayant tort c'est donc que l'inverse est vrai et \robbe{} ne saurait avoir tort. 
\begin{quote}
    chaque fois que la rumeur publique, ou tel critique spécialisé qui tout à la fois la reflète et l’alimente, nous prête une intention, on peut affirmer sans gros risque d’erreur que nous avons exactement l’intention inverse\footnote{\op143}.
\end{quote}
Cependant, puisqu'il s'agit pour \robbe{} de rétablir les faits contre les préjugés, il ne suffit en réalité pas d'en dire le contraire. C'est là qu'est l'habileté et la distinction entre «~l'exact contre-pied\footnote{\op144}~» et l'«~exactement [...] inverse\footnote{\op143}~»~: les positions adverses résumés portent sur des positions de principes aux contours peu défini mais trop généraux pour ne pas prêter aisément le flan critique~; ainsi il suffit de proposer une position plus ouverte en contre point, pour sembler plus difficilement réfutable. Ainsi à la proposition «~Le Nouveau Roman a fait table rase du passé~» il rétorque «~Le Nouveau Roman ne fait que poursuivre une évolution constante du genre romanesque~» et non pas ce qui pourrait en être l'exact inverse «~Le Nouveau Roman épouse pleinement l'ensemble de la tradition littéraire.~».

Admettons que la technique est appropriée pour un auteur qui souhaite défendre la liberté perçue comme un refus de supposées règles édictées par une tradition réactionnaire~: plus la préconisation sera souple mieux elle s'intégrera au sein du système, dont on pourra toujours dire, sans doute à raison, qu'il n'est pas tout à fait un système.

%De par sa structure, ses arguments à la fois rhétoriques mais également théoriques (où ce que l'on serait tenté d'appeler «~la bonne foi~» est l'un des concepts opérant la distinction), \robbe{} se place comme le tenant du bon sens. 
%mauvaise foi des critiques, difficile à démontrer autant laisser béton






%Nous proposons ici un examen des moyens rhétoriques mais surtout du rapport complexe qu'entretiennent la «~vérité~», du moins la signification, et le réel articulés et explicités dans l'œuvre par le recourt au «~bon sens~». 





\subsection{Des dichotomies structurantes}

Ces techniques qui servent l'argumentation sont sous-tendues et mises en mouvement par un jeu d'oppositions frontales~: et si à chaque proposition des adversaires, il n'y a pas toujours une réponse terme à terme~; le lecteur décèle une axiologie sous-jacente qui organise (peut-être plus qu'elle ne résulte de) l'ensemble de la démonstration. Cette axiologie est d'autant plus perceptible que les termes qui la composent sont clairement identifiés.

Au sein de \punr les termes qui constituent selon Robbe-Grillet le débat sont l'objet même du débat. Ainsi s'ouvre «~Sur quelques notions périmées~» à la page~29~:
\begin{quote}
    La critique traditionnelle a son vocabulaire. Bien qu’elle se défende beaucoup de porter sur la littérature des jugements systématiques (prétendant, au contraire, aimer librement telle ou telle œuvre d’après des critères «~naturels~»~: le bon sens, le cœur, etc.), il suffit de lire avec un peu d’attention ses analyses pour voir aussitôt paraître un réseau de mots-clefs, trahissant bel et bien un système.

    Mais nous sommes tellement habitués à entendre parler de «~personnage~», d’«~atmosphère~», de «~forme~» et de «~contenu~», de «~message~», du «~talent de conteur~» des «~vrais romanciers~», qu’il nous faut un effort pour nous dégager de cette toile d’araignée et pour comprendre qu’elle représente une idée sur le roman (idée toute faite, que chacun admet sans discussion, donc idée morte), et point du tout cette prétendue «~nature~» du roman en quoi l’on voudrait nous faire croire.
\end{quote}
Il s'agit également pour Robbe-Grillet de faire accepter à ses adversaires, ou plutôt à ses lecteurs, le fait que ce sont bien les termes qui sont en débat~: que «~personnage~» et «~histoire~» ne sont pas des évidences. Leur emploi ainsi mis en valeur, le signe est mis en valeur et son emploi impropre souligné par l'argumentaire, telle à la page~48~:
\begin{quote}
    Il est dès lors normal que l'accusation de «~formalisme~» soit l'une des plus grave dans la bouche de nos censeurs des deux bords.
\end{quote}

Explicité dans la lettre du texte, le procédé l'est avant tout explicité par des choix stylistiques qui relèvent de la typographie. De manière quasi-systématique, les notions adverses qui se superposent souvent efficacement aux notions périmées, sont encadrées de guillemets, alors que des expressions privilégiées par Robbe-Grillet sont mises en valeur par l'italique.

Ce procédé est courant et s'inscrit dans les conventions habituelles des essais, mais chez Robbe-Grillet cet usage prend une place prépondérante et structure l'ensemble du recueil. Nous nous sommes efforcés au sein de notre édition numérique de produire un inventaire sous la forme de deux index de ces notions adverses et expressions privilégiées (reproduits en annexe \textit{infra} voir~: \ref{gloss}).

Notons que l'appréciation «~privilégiées~» porte bien sur la formulation et non le contenu sémantique~: «~Aussi le livre n’est-il pas écrit dans un langage [...]. Seuls, en effet, les objets déjà chargés d’un contenu humain flagrant sont neutralisés, avec soin, et \textit{pour des raisons morales}~» page~69, n'est pas privilégié le fait de choisir son style pour des raisons morales mais bien le propos de Robbe-Grillet sur le style de \textit{L'Étranger}, et ce, peu importe le statut axiologique de ce style au sein de \punr{}.

Précisons que tous les termes entre guillemets ne sont pas nécessairement considérés comme étant l'une de ces «~notions adverses~», les guillemets marque également les citations extraites d'ouvrages ou de propos fictif prêtés à des personnages invoqués pour les besoins de l'argumentation (voir \textit{supra}~: \ref{histoireDispute}). Cependant les notions adverses, mise entre guillemet se superposent bel et bien à des citations bien qu'il ne fasse aucun doute qu'il s'agit là du propos de l'auteur, ce dernier insiste sur sa répugnance à les employer et se dégage de la responsabilité de cette partie de l'énnoncé. 

Les uns étant l'exact opposé des autres~: les expressions privilégiées sont des signifiants mis en valeur pour leur qualité indépendamment de l'approbation ou non de l'auteur sur leur signifié~; les notions adverses sont des signifiants dénigrés au nom d'une relation entre ce signifiant et son signifié fautive (on ne devrait pas dire «~l'humain~» ou «~l'inhumain\footnote{\op 56-58}~» car le propos est faux).


Il n'en va pas tout à fait de même pour les termes en italique. En effet, si l'intuition d'origine avait pour origine les mises en valeurs typographiques de certains termes, mis entre guillemets sans être des citations ou mis en italique, il semblait évident que tous les termes entre guillemets ne méritaient pas d'être intégrés en tant que «~notions adverses~» et tous les termes en italiques en «~expressions privilégiées~»~; certains étaient des citations d'autre un simple procédé d'emphase.

Or, à bien y regarder les expressions privilégiées n'étaient-elles pas toutes des effets d'emphase~? Ne voit-on pas plutôt Robbe-Grillet insister sur telle ou telle formulation par opposition à une autre explicitement mentionnée ou non~? 


C'est en tout cas ce qui nous est apparu après un examen minutieux. En effet, les expressions privilégiées loin de revêtir l'évidence de concepts nettement définis sont davantage des adjectifs qui changent le sens des termes qui les entourent et font signe vers les concepts d'une philosophie, du moins d'une esthétique qui se définit surtout par son opposition à la métaphysique éculée contre laquelle Robbe-Grillet engage son texte. Ainsi les articles indéfinis «~\textit{des} questions, et \textit{des} réponses~» page~64, l'adjectif «~vrai~» pages~24 et~162 (3~fois) et les multiples occurences du verbes «~être~» souvent accompagné de l'adverbe «~là~» (pages~20, 21, 177), expressions privilégiées contenant une assertion positive sont-ils parmi les termes les plus courants de la langue française et, seul, difficilement rattachable à telle ou telle théorie ésthétique, si ce n'est à une insistance sur l'immanence opposée à la métaphysique prêté aux adversaire de \robbe{}. Sans doute la relative banalité des termes mis en avant par la typographie sert à la démonstration~: l'acceptation de l'immanence des choses difficilement contestable se trouve aussi dans ce choix typographique qui sert à démontrer avec force que l'évidence, c'est-à-dire le bon sens est du côté de Robbe-Grillet.





%Cet emploi des signes typographiques explicité par le projet du troisième article du recueil (les notions périmées) contamine l'ensemble du recueil

De même si les oppositions ne sont pas systématiques (un terme en italique ne répond pas toujours à une expression entre guillemet) le jeu d'opposition est suffisamment soutenu tout au long du recueil pour jouer sur l'implicite et laisser au lecteur le soin d'investir les dichotomies, les faisant siennes au passage. Ainsi l'emploi de l'expression «~\textit{vraie} ville\footnote{\op163}~» pour désigner Istanbul dans \textit{L'Immortelle} induit \textit{via} le détour implicite par l'expression antithétique «~une \textit{fausse} ville~», le dénigrement du commentaire des critiques sur la représentation qui en est faite, tant il paraît ridicule qu'Istanbul serait une \textit{fausse} ville, \textit{a fortiori} dans une œuvre cinématographique.

%est ridiculisée par sa mise en opposition avec l'expression «~fausse mer~» 


%Si nous ne pourrons tout à fait éviter d'expliciter les liens qui régissent et organisent les termes de l'axiologie de \punr, nous nous emploierons davantage ici à montrer le jeu stylistique qui les met en valeur.


%le jeu d'opposition bien que réel est aussi en partie illusoire puisque dès l'exposition de la th_se adverse elle est déjà discrédité

%À quoi cela sert-il, dira-t-on, si c’est pour aboutir ensuite, après un temps plus ou moins long, à un nouveau formalisme, aussi sclérosé bientôt que ne l’était l’ancien ? Cela revient à demander pourquoi vivre, puisqu’il faut mourir et laisser la place à d’autres vivants. L’art est vie. Rien n’y est jamais gagné de façon définitive. Il ne peut exister sans cette remise en question permanente. Mais le mouvement de ces évolutions et révolutions fait sa perpétuelle renaissance. p. 172
% rappele la tragédie mais n'est pas vécu comme un drame loin de là

%p. 129-130 sur Nr HN version originale et constitution du collectif
Enfin rappelons que le débat sur les termes semblent l'un des enjeux (sinon le seul) du recueil ces expressions privilégiées contre des notions adverses induisent un positionnement esthétique profondément inscrit dans son époque, contre Sartre (alors même qu'il reprend certain de ses termes) et pour une mesure de la valeur littéraire en fonction de l'écart qu'une œuvre creuse entre ses principes et ceux de l'habitude.


%TRANSITION LEUR PROXIMITÉ AVEC SARTRE
\newpage

\section{Une théorie esthétique}

\subsection{De plein pieds dans son époque : postulats et références}
\label{3phéno}


%phénoménologie, critique du sujet de la rhétorique 
    %du romantisme
    %plus original contre la mélancolie après les camps



Notons un trait, qui nous paraît d'époque~: les «~critiques bourgeois~» mentionnés page~47 comme élément portant la charge péjorative d'une comparaison~; tant il est évident qu'on ne peut vouloir être bourgeois. On pourrait identifier les critiques désignés ainsi aux Académiciens (p.~%TROUVE
) et aux critiques traditionnels (p.~%TROUVE
) mais il convient de souligner le fait que le terme «~bourgeois~» n'apparaît chez Robbe-Grillet que pour dénigrer «~les engagés~» en les qualifiant de ce que l'auteur considère sans doute comme le plus grand affront qu'on puisse leur faire. Au contraire lorsque Robbe-Grillet s'attaque à la littérature traditionnelle ou académique, il évite ce terme~: son désaccord ne se situe pas dans le champs social mais bien dans celui de la littérature et à ce titre il semble incarner une position peut-être plus radicale que les socialistes réalistes.


\subsection{Un positionnement littéraire, contre Sartre ?}
\label{vsSartre}
%rapport ambigue qui semble bien davntage un positionnement dans le champs littéraire et un apport de nuance que l'on ne pourrait balayer d'un revers de mains comme n'étant qu'une posture


%proximité des termes en italique parfois avec Sartre


%p. 50-51
%fin du recueil 

%p. 101-102 GY sur les reprises à Barthes y a aussi e^tre là qui d'apr_s Morissette (et moi aussi) viendrait de Sartre qui popularise le vocable de Hgger + dans la nausée c'est comme ça aussi

%morissette souligne que être là est terminologie popularisée par Sartre

\subsection{Une théorie de l'art : l'écart}
\label{theorie}
    % Innovation contre l'habitude
    % ne fait pas référence à horizon d'attente mais on n'y pense
    % antiroman ?
    % dès lors la moindre divergence paraît un élément important + les commencements

%\subsection{Quelle(s) postérité(s) à \punr?}
%\label{posterite}
    


\newpage
\addcontentsline{toc}{section}{Conclusion}
%plongée dans une époque, mais aussi et surtout percevoir l'enthousiasme du moins l'énergie des commencements


\newpage

\section{Annexes}

\subsection{Index rhétoriques}
\label{gloss}
\subsubsection{Index des notions adverses}

\begin{itemize}

    \item écrire pour le «~grand public~»{\color{gray}~; [8]}

    \item auteur «~difficile~»{\color{gray}~; [8]}

    \item cette récidive~–~qualifiée de «~manifeste~»{\color{gray}~; [8]}

    \item nouvelle «~école~» romanesque{\color{gray}~; [9]}

    \item «~École du regard~»{\color{gray}~; [9]}

    \item «~Roman objectif~»{\color{gray}~; [9]}

    \item «~École de Minuit~»{\color{gray}~; [9]}

    \item le grand romancier, le «~génie~»{\color{gray}~; [11]}

    \item «~messages~» que seul le lecteur doit déchiffrer{\color{gray}~; [11]}

    \item son «~intelligence~» ne lui est plus d’aucun secours{\color{gray}~; [14]}

    \item «~bon~» roman{\color{gray}~; [17]}

    \item le «~cœur~» humain{\color{gray}~; [18]}

    \item paysage est «~austère~» ou «~calme~»{\color{gray}~; [21]}

    \item nous pensons aussitôt~: «~C’est de la littérature~»{\color{gray}~; [21]}

    \item cet univers des «~significations~» (psychologiques, sociales, fonctionnelles){\color{gray}~; [23]}

    \item «~le cœur romantique des choses~»{\color{gray}~; [24]}

    \item vieux mythes de la «~profondeur~»{\color{gray}~; [26]}

    \item  l’idée de «~condition~» remplaçant désormais celle de «~nature~»{\color{gray}~; [27]}

    \item tous les «~au-delà~» de la métaphysique{\color{gray}~; [27]}

    \item des critères «~naturels~»{\color{gray}~; [29]}

    \item «~personnage~»{\color{gray}~; [29]}

    \item «~atmosphère~»{\color{gray}~; [29]}

    \item «~forme~»{\color{gray}~; [29]}

    \item «~contenu~»{\color{gray}~; [29]}

    \item «~message~»{\color{gray}~; [29]}

    \item «~talent de conteur~»{\color{gray}~; [29]}

    \item «~vrais romanciers~»{\color{gray}~; [29]}

    \item «~nature~»{\color{gray}~; [29]}

    \item «~avant-garde~»{\color{gray}~; [30]}

    \item l’étiquette~: «~avant-garde~»{\color{gray}~; [30]}

    \item mots magiques~: «~avant-garde~»{\color{gray}~; [30]}

    \item «~laboratoire~»{\color{gray}~; [31]}

    \item «~anti-roman~»{\color{gray}~; [31]}

    \item «~personnage~»{\color{gray}~; [31]}

    \item «~vrai~» romancier{\color{gray}~; [31]}

    \item «~il crée des personnages~»{\color{gray}~; [31]}

    \item posséder un «~caractère~»{\color{gray}~; [31]}

    \item culte exclusif de «~l’humain~»{\color{gray}~; [33]}

    \item une «~histoire~»{\color{gray}~; [34]}

    \item sait «~raconter une histoire~»{\color{gray}~; [34]}

    \item on n’est pas en train de lui «~raconter des histoires~»{\color{gray}~; [35]}

    \item notion de «~tranche de vie~»{\color{gray}~; [36]}

    \item quelque chose de «~naturel~»{\color{gray}~; [36]}

    \item une «~action~»{\color{gray}~; [38]}

    \item «~l’engagement~»{\color{gray}~; [40]}

    \item les accusations de «~décadence~», de «~gratuité~», de «~formalisme~»{\color{gray}~; [41]}

    \item cesser de craindre «~l’art pour l’art~»{\color{gray}~; [43]}

    \item les «~valeurs~» terrestres de la société bourgeoise{\color{gray}~; [44]}

    \item tout «~au-delà~» spirituel de notre monde visible{\color{gray}~; [44]}

    \item l’expression la plus «~bourgeoise~»{\color{gray}~; [46]}

    \item les recherches dites «~de laboratoire~»{\color{gray}~; [46]}

    \item séparer la «~forme~» d’un roman de son «~contenu~»{\color{gray}~; [47]}

    \item le «~grand~» roman{\color{gray}~; [47]}

    \item accusation de «~formalisme~»{\color{gray}~; [48]}

    \item leur «~signification profonde~», c’est-à-dire leur contenu{\color{gray}~; [49]}

    \item les romans «~engagés~»{\color{gray}~; [50]}

    \item est «~responsable~»{\color{gray}~; [50]}

    \item le «~message~» de l’auteur{\color{gray}~; [51]}

    \item «~Untel a quelque chose à dire et il le dit bien.~»{\color{gray}~; [51]}

    \item le reproche de «~gratuité~»{\color{gray}~; [51]}

    \item terme de «~formalisme~»{\color{gray}~; [52]}

    \item leur «~contenu~»{\color{gray}~; [52]}

    \item «~surfaces~»{\color{gray}~; [56]}

    \item des «~valeurs~»{\color{gray}~; [56]}

    \item son rôle «~naturel~»{\color{gray}~; [56]}

    \item au nom de l’«~humain~»{\color{gray}~; [56]}

    \item une œuvre «~inhumaine~»{\color{gray}~; [58]}

    \item En tant que «~personnage~»{\color{gray}~; [58]}

    \item le temps est «~capricieux~»{\color{gray}~; [59]}

    \item la montagne «~majestueuse~»{\color{gray}~; [59]}

    \item du «~cœur~» de la forêt{\color{gray}~; [59]}

    \item un soleil «~impitoyable~»{\color{gray}~; [59]}

    \item un village «~blotti~»{\color{gray}~; [59]}

    \item le mot «~blotti~»{\color{gray}~; [60]}

    \item verbe «~se blottir~»{\color{gray}~; [61]}

    \item le spectacle moral de la «~majesté~»{\color{gray}~; [61]}

    \item considérés comme la réalité profonde de l’univers matériel{\color{gray}~; [62]}

    \item l’idée d’une nature{\color{gray}~; [63]}

    \item une essence commune pour toute la «~création~»{\color{gray}~; [63]}

    \item  notre prétendue «~nature~»{\color{gray}~; [64]}

    \item «~galop~» d’un nuage, ou de sa «~crinière échevelée~»{\color{gray}~; [64]}

    \item la crinière d’un étalon «~lance des flèches~»{\color{gray}~; [64]}

    \item tragédie{\color{gray}~; [66]}

    \item fonctionnement de la «~solitude~»{\color{gray}~; [67]}

    \item le «~pont d’âme~» subsiste entre elles et nous{\color{gray}~; [69]}

    \item le domaine d’élection de la tragédie soit le «~romanesque~»{\color{gray}~; [69]}

    \item le bon «~personnage~» de roman{\color{gray}~; [69]}

    \item L’intrigue sera d’autant plus «~humaine~» qu’elle sera plus équivoque.{\color{gray}~; [69]}

    \item des idées de «~nature~»{\color{gray}~; [69]}

    \item le germe «~tragique~»{\color{gray}~; [70]}

    \item Le regard demeure malgré tout notre meilleure arme, surtout s’il s’en tient aux seules
                  lignes. Quant à sa «~subjectivité~»{\color{gray}~; [82]}

    \item les mots «~théâtre de laboratoire~»{\color{gray}~; [122]}

    \item les «~héros~» de Beckett{\color{gray}~; [125]}

    \item on y «~pensait~» ferme{\color{gray}~; [126]}

    \item le «~Nouveau Roman~»{\color{gray}~; [143]}

    \item les «~écoles~»{\color{gray}~; [145]}

    \item le «~vrai roman~»{\color{gray}~; [146]}

    \item la «~confusion~» dans les descriptions{\color{gray}~; [146]}

    \item de «~personnages~»{\color{gray}~; [147]}

    \item plus «~humains~» que les nôtres{\color{gray}~; [148]}

    \item mot «~objectivité~»{\color{gray}~; [148]}

    \item leur contribution «~théorique~»{\color{gray}~; [157]}

    \item le monde «~réel~»{\color{gray}~; [158]}

    \item notions si peu visuelles de «~droite~» et de «~gauche~»{\color{gray}~; [160]}

    \item le manque de «~naturel~» dans le jeu des acteurs{\color{gray}~; [162]}

    \item ce qui est «~réel~»{\color{gray}~; [162]}

    \item «~cartes-postales~»{\color{gray}~; [163]}

    \item «~réalisme~»{\color{gray}~; [163]}

    \item une «~histoire vécue~»{\color{gray}~; [163]}

    \item le «~personnage~» principal{\color{gray}~; [164]}

    \item de «~temps~»{\color{gray}~; [164]}

    \item plus «~réelle~»{\color{gray}~; [165]}

    \item le seul «~personnage~» important{\color{gray}~; [166]}

    \item «~destin~»{\color{gray}~; [168]}

    \item une «~histoire~» au sens traditionnel{\color{gray}~; [169]}

    \item de créer du «~réel~»{\color{gray}~; [171]}

    \item aussi bien partagé que le «~bon sens~»{\color{gray}~; [172]}

    \item Il ne s’agit pas de «~faire mieux~»{\color{gray}~; [172]}

    \item au «~progrès~» de nos connaissances physiques{\color{gray}~; [173]}

    \item le «~bon outil~» dont la critique soviétique{\color{gray}~; [174]}

    \item mot «~réalisme~»{\color{gray}~; [174]}

    \item le rôle de ce dernier se limite à «~explorer~» et à «~exprimer~» la réalité{\color{gray}~; [174]}

    \item plus «~réalistes~»{\color{gray}~; [175]}

    \item observer les choses «~sur le vif~» et de me «~rafraîchir la mémoire~»{\color{gray}~; [176]}

    \item le «~vraisemblable~» et le «~conforme au type~»{\color{gray}~; [177]}

    \item des «~personnages~»{\color{gray}~; [177]}

    \item Le petit détail qui «~fait vrai~»{\color{gray}~; [177]}

    \item sens «~profond~»{\color{gray}~; [179]}

    \item notion théologique de «~grâce~»{\color{gray}~; [179]}

    \item évoquerait le «~réel~»{\color{gray}~; [179]}

    \item le «~Nouveau Roman~»{\color{gray}~; [182]}

    \item une «~mode qui passe~»{\color{gray}~; [182]}

    \item le roman «~éternel~»{\color{gray}~; [183]}
\end{itemize}



\subsubsection{Index des expressions privilégiées}
% entre les deux  
    
    
\begin{itemize}

    \item littérature entièrement \textit{nouvelle}{\color{gray}~; [17]}

    \item la \textit{liberté}{\color{gray}~; [21]}

    \item Il \textit{est}, tout simplement{\color{gray}~; [21]}

    \item les choses \textit{sont là.}{\color{gray}~; [22]}

    \item maintenant on \textit{voit} la chaise{\color{gray}~; [22]}

    \item l’image a restitué d’un seul coup (sans le vouloir) leur \textit{réalité.}{\color{gray}~; [23]}

    \item le caractère \textit{inhabituel} du monde{\color{gray}~; [23]}

    \item leur \textit{présence} que les objets et les gestes{\color{gray}~; [23]}

    \item gestes et objets seront \textit{là}{\color{gray}~; [23]}

    \item avant d’être \textit{quelque chose}{\color{gray}~; [23]}

    \item un \textit{ailleurs} immatériel et instable{\color{gray}~; [24]}

    \item y a rien d’autre de \textit{vrai.}{\color{gray}~; [25]}

    \item créer que \textit{pour rien}{\color{gray}~; [42]}

    \item la chose \textit{la plus importante au monde.}{\color{gray}~; [43]}

    \item œuvre créée \textit{pour } l’expression d’un contenu{\color{gray}~; [43]}

    \item être seulement \textit{ce qu’ils sont.}{\color{gray}~; [44]}

    \item opposer l’\textit{écriture }{\color{gray}~; [47]}

    \item L’œuvre d’art, comme le monde, est une forme vivante~: elle \textit{est,}{\color{gray}~; [49]}

    \item la \textit{nécessité,} à quoi l’œuvre d’art se reconnaît{\color{gray}~; [51]}

    \item mais nécessaire \textit{pour rien~;}{\color{gray}~; [52]}

    \item appeler l’\textit{aliénation} de la littérature{\color{gray}~; [52]}

    \item précisément de \textit{tout} récupérer{\color{gray}~; [57]}

    \item ce terme d’\textit{humain}{\color{gray}~; [57]}

    \item sa prétendue \textit{signification,}{\color{gray}~; [58]}

    \item le point de vue \textit{humaniste,}{\color{gray}~; [59]}

    \item le village à être seulement «~situé~»{\color{gray}~; [60]}

    \item Le mot «~blotti~»{\color{gray}~; [60]}

    \item ne reste jamais entièrement \textit{extérieur.}{\color{gray}~; [61]}

    \item ce paysage existait \textit{avant }moi{\color{gray}~; [63]}

    \item vraiment \textit{lui}{\color{gray}~; [63]}

    \item il l’était \textit{déjà }avant moi{\color{gray}~; [63]}

    \item  repousser l’idée «~pananthropique~»{\color{gray}~; [64]}

    \item \textit{Toutes} les analogies{\color{gray}~; [64]}

    \item nature commune à toutes choses, c’est-à-dire \textit{supérieure.}{\color{gray}~; [65]}

    \item la \textit{seule question }de notre civilisation gréco-chrétienne{\color{gray}~; [65]}

    \item \textit{des} questions, et \textit{des} réponses{\color{gray}~; [65]}

    \item il \textit{pourra,} du moins, un jour.{\color{gray}~; [65]}

    \item une communion, mais \textit{douloureuse,}{\color{gray}~; [66]}

    \item un \textit{envers,} c’est un piège{\color{gray}~; [66]}

    \item un \textit{vrai} silence{\color{gray}~; [67]}

    \item une \textit{distance intérieure,}{\color{gray}~; [69]}

    \item être \textit{double.}{\color{gray}~; [69]}

    \item nous mène \textit{d’abord} à la tragédie{\color{gray}~; [69]}

    \item établir entre eux un autre rapport que d’\textit{étrangeté.}{\color{gray}~; [70]}

    \item langage aussi \textit{lavé}{\color{gray}~; [70]}

    \item et \textit{pour des raisons morales}{\color{gray}~; [70]}

    \item couleur est changeante, donc elle \textit{vit}{\color{gray}~; [74]}

    \item choses sont vivantes, \textit{comme lui-même.}{\color{gray}~; [74]}

    \item il parle \textit{pour }les choses, \textit{avec} elles, dans leur \textit{cœur,}{\color{gray}~; [77]}

    \item telle \textit{réflexion,} chez Ponge{\color{gray}~; [77]}

    \item \textit{n’étant pas l’homme,} elles restent constamment hors d’atteinte{\color{gray}~; [78]}

    \item connaître l’\textit{intérieur} des choses{\color{gray}~; [79]}

    \item ce qu’il y a \textit{dans} ces choses{\color{gray}~; [79]}

    \item  poursuivent la \textit{connaissance} des textures{\color{gray}~; [79]}

    \item \textit{en face} des choses.{\color{gray}~; [80]}

    \item tel que l’oriente \textit{mon point de vue~;}{\color{gray}~; [82]}

    \item sert précisément à définir \textit{ma situation dans le monde.}{\color{gray}~; [82]}

    \item identifier ce qui ne l’est pas, ce qui \textit{est un,}{\color{gray}~; [83]}

    \item ce malheur est \textit{situé} dans l’espace et le temps{\color{gray}~; [83]}

    \item Les mots et les phrases, par exemple, y deviennent aussi des \textit{objets, }{\color{gray}~; [109]}

    \item fragment détaché s’était \textit{éternisé à l’état de chute}{\color{gray}~; [109]}

    \item \textit{se réfugier} dans le rêve{\color{gray}~; [110]}

    \item la clef de \textit{ce monde-ci}{\color{gray}~; [110]}

    \item ce \textit{rêve éveillé} pourrait simplement être \textit{l’art,}{\color{gray}~; [110]}

    \item la vue d’\textit{immobile}{\color{gray}~; [111]}

    \item elle \textit{doit} l’être{\color{gray}~; [111]}

    \item Sa situation \textit{dans} le monde{\color{gray}~; [112]}

    \item notre condition de \textit{mortels}{\color{gray}~; [112]}

    \item ou plutôt \textit{seront} son œuvre{\color{gray}~; [113]}

    \item le corps de la parole et de l’écriture, \textit{le langage. }{\color{gray}~; [116]}

    \item ils n’ont rien \textit{à inventer}{\color{gray}~; [129]}

    \item cet \textit{ici} inéluctable, répond un éternel \textit{maintenant~:}{\color{gray}~; [132]}

    \item toute idée de \textit{progrès} qu’une quelconque \textit{signification.}{\color{gray}~; [132]}

    \item \textit{Il n’y a plus de présent,}{\color{gray}~; [133]}

    \item des \textit{choses,}{\color{gray}~; [148]}

    \item \textit{toujours} dans une aventure{\color{gray}~; [149]}

    \item \textit{un homme} qui voit{\color{gray}~; [149]}

    \item n’est \textit{jamais} chronologique{\color{gray}~; [149]}

    \item une \textit{vraie} ville{\color{gray}~; [163]}

    \item une \textit{vraie} femme{\color{gray}~; [163]}

    \item  Le \textit{vrai,} le \textit{faux} et le \textit{faire croire}{\color{gray}~; [163]}

    \item le \textit{peu de réalité,}{\color{gray}~; [163]}

    \item plus d’\textit{ailleurs} possible que d’\textit{autrefois.}{\color{gray}~; [166]}

    \item c’est \textit{dans sa tête} que se déroule{\color{gray}~; [166]}

    \item \textit{imaginée }par lui{\color{gray}~; [166]}

    \item \textit{déprend} du piège{\color{gray}~; [168]}

    \item conscient, \textit{créateur.}{\color{gray}~; [169]}

    \item elle \textit{constitue} la réalité{\color{gray}~; [175]}

    \item l’\textit{illusion réaliste.}{\color{gray}~; [176]}

    \item plus réelles, \textit{parce qu}’elles étaient maintenant imaginaires{\color{gray}~; [176]}

    \item le \textit{faux}{\color{gray}~; [177]}

    \item de \textit{vérisme.}{\color{gray}~; [177]}

    \item leur \textit{vraisemblance}{\color{gray}~; [178]}

    \item ce qu’on nomme l’\textit{absurde~?}{\color{gray}~; [178]}

    \item Cela \textit{est,} et c’est tout{\color{gray}~; [178]}

    \item auteur \textit{réaliste}{\color{gray}~; [178]}

    \item parler d’\textit{autre chose.}{\color{gray}~; [179]}

    \item en vérité \textit{tout change sans cesse}{\color{gray}~; [183]}

    \item il y a \textit{toujours du nouveau}{\color{gray}~; [183]}
\end{itemize}
\newpage
\tableofcontents
\end{document}