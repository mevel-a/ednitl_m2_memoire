\documentclass[12pt, a4paper]{article}
\usepackage[utf8]{inputenc}
\usepackage[T1]{fontenc}
\usepackage{lmodern}
\usepackage[dvipsnames]{xcolor}
\usepackage{fancyhdr}
\usepackage{reledmac}
\usepackage{float}
\usepackage{graphicx}
\usepackage[top=2cm, bottom=2cm, left=4cm, right=4cm, heightrounded, marginparwidth=3.5cm, marginparsep=0.3cm]{geometry}
\renewcommand{\headrulewidth}{0.1pt}
\renewcommand{\footrulewidth}{0.3pt}
\usepackage{hyperref}
\pagestyle{fancy}
\lhead{{\scshape{Mével}} Adrien M2 EdNitl}
\chead{}
\rhead{2022-2023}
\fancyfoot[]{}
\rfoot{\thepage}
\usepackage[french]{babel}
\setlength{\headheight}{20.61049pt}

\begin{document}
\vspace*{3cm}


\begin{center}
\textsc{Projet de recherche, revu}
\end{center}
\vspace{3cm}

\section{Introduction}
\small Ce qui suit correspond à la reprise de mes notes après le premier entretien avec Mme~\textsc{de~Challonge}, le 12~octobre~2022 à 11 heures (il est 12:17).

\vspace{1cm}
\normalsize
Après avoir pensé réduire la voilure sur le projet manifeste en abordant pourquoi pas le surréalisme/oulipo/Nr mais n'en prendre que deux et les difficultés à produire quelque chose d'intéressant dans ce cadre qui permetrait de ne pas produire un travail de thèse (en gros). Nous abordons l'idée de comparer \textit{Pour un nouveau roman} et \textit{L'ère du soupçon} texte de 56 et 63, le projet finit par prendre forme.

\section{Sujet, proprement dit}
Produire une édition numérique de pour un nouveau roman.
Plus facile car plus explicite et de nombreux articles dessus que Sarraute.

L'édition critique et numérique ayant vocation à être une sorte de grille de lecture.
\begin{itemize}
    \item Notions importantes
    \item Enjeux --> versant scientifique
    \item Introduction du texte
    \item La forme --> versant scientifique
    \item Développement ++ en note tout en prenant modèle de concision de la Pléiade.
    \item Notices
    \item notes
    \item Glossaire
\end{itemize}

L'édition critique faisant double emploi de versant littéraire de la recherche.


\section{Modèles possibles}

En GF poche \href{https://www.lalibrairie.com/livres/lettres-a-madame-la-marquise-sur-le-sujet-de-la-princesse-de-cleves_0-60674_9782080711144.html}{Théorie de Valincourt, dans ses lettres réagissant à la Princesse de Clève}

Transposer ce genre de chose au XXe.

Ou la pléiade~:
Perec Les Choses, Homme qui dort Vol. I par Mme~\textsc{de~Challonge}

Duras (7 différents) vol. II III IV, par Mme~\textsc{de~Challonge}

Sarraute soupçon


\section{Génétique ?}
Bibliothèque d'ardennes ? En normandie ?
Il faudrait pour cela  trouver des brouillons dans des fonds d'archives contemporaine. 

au demeurant très intéressant puisque cela ouvre la possibilité de faire des variantes.

BNF 
Description des fonds en ligne peut-être y aurait-il des brouillons de RG~?




\section{To do}
Contacter Marchal pour lui demander si c'est ok (le versant scientifique est ok dixit de Challonge) 



\end{document}