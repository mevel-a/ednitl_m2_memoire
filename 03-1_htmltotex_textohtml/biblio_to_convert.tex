\section{Bibliographie}
	
		
		\vspace*{2cm}
		\setlength{\parindent}{0cm}
{\large\textsc{Édition citée de Pour un nouveau roman}}
		\vspace*{1cm}
		\setlength{\parindent}{25pt}
		
\textsc{Robbe-Grillet}~Alain, \textit{Pour un nouveau roman}, Paris, Les Éditions de Minuit, coll.~«~Double~», 2013 [1963]\par
		\vspace*{2cm}
		\setlength{\parindent}{0cm}
{\large\textsc{Première publication des textes constituants l'ensemble}}
		\vspace*{1cm}
		\setlength{\parindent}{25pt}
		
		
		

		
	\textsc{Robbe-Grillet}~Alain, «~Il écrit comme Stendhal~», Paris, \textit{L'Express, }, 1955/10/25, p.~8\par
	\textsc{Robbe-Grillet}~Alain, «~La littérature aujourd'hui - VI~», Paris, \textit{Tel Quel, n°~14}, 1963 été, p.~39-45\par
	\textsc{Robbe-Grillet}~Alain, «~Une voie pour le roman futur~», Paris, \textit{Nouvelle Revue Française, n°~43}, 1956/07, p.~77-84\par
	\textsc{Robbe-Grillet}~Alain, «~Pour un réalisme de la présence~», Paris, \textit{L'Express}, 1956/01/17, p.~11\par
	\textsc{Robbe-Grillet}~Alain, «~Réalisme et révolution~», Paris, \textit{L'Express}, 1955/01/03, p.~15\par
	\textsc{Robbe-Grillet}~Alain, «~Littérature engagée, littérature réactionnaire~», Paris, \textit{L'Express}, 1955/12/20, p.~11\par
	\textsc{Robbe-Grillet}~Alain, «~La Forme et le contenu~», Paris, \textit{France Observateur, n°~392}, 1957/11/14, p.~19\par
	\textsc{Robbe-Grillet}~Alain, «~Il n'y a pas "d'avant garde"~», Paris, \textit{France Observateur, n°~388}, 1957/10/17, p.~19\par
	\textsc{Robbe-Grillet}~Alain, «~Un joli talent de conteur~», Paris, \textit{France Observateur, n°~390}, 1957/10/31, p.~19\par
	\textsc{Robbe-Grillet}~Alain, «~Écrire pour son temps~», Paris, \textit{France Observateur, n°~387}, 1957/10/10, p.~17\par
	\textsc{Robbe-Grillet}~Alain, «~Le réalisme socialiste est bourgeois~», Paris, \textit{L'Express}, 1956/02/21, p.~11\par
	\textsc{Robbe-Grillet}~Alain, «~La mort du personnage~», Paris, \textit{France Observateur, n°~389}, 1957/10/24, p.~20\par
	\textsc{Robbe-Grillet}~Alain, «~Nature, Humanisme, Tragédie~», Paris, \textit{Nouvelle Revue Française, n°~70}, 1958/10, p.~580-603\par
	\textsc{Robbe-Grillet}~Alain, «~Énigmes et transparences chez Raymond Roussel~», Paris, \textit{Critique, n°~199}, 1963/12, p.~1027-1033\par
	\textsc{Robbe-Grillet}~Alain, «~La conscience malade de Zeno~», Paris, \textit{Nouvelle Revue Française, n°~19}, 1954/07, p.~138-141\par
	\textsc{Robbe-Grillet}~Alain, «~Joë Bousquet le rêveur~», Paris, \textit{Critique, n°~77}, 1953/10, p.~819-829\par
	\textsc{Robbe-Grillet}~Alain, «~Samuel Beckett ou la présence sur la scène~», Paris, \textit{Critique, n°~189}, 1963/02, p.~108-114\par
	\textsc{Robbe-Grillet}~Alain, «~Samuel Beckett, Auteur dramatique~», Paris, \textit{Critique, n°~69}, 1953/02, p.~108-114\par
	\textsc{Robbe-Grillet}~Alain, «~Un roman qui s'invente lui-même~», Paris, \textit{Critique, n°~80}, 1954/01, p.~82-88\par
	\textsc{Robbe-Grillet}~Alain, «~Nouveau roman, homme nouveau~», Paris, \textit{Revue de Paris, n°~68}, 1961/09, p.~115-121\par
	\textsc{Robbe-Grillet}~Alain, «~Comment mesurer l'inventeur des mesures~», Paris, \textit{L'Express, n°~627}, 1963/06/20, p.~44-45\par
	\textsc{Robbe-Grillet}~Alain, «~Monsieur Personne répond... Pour un "nouveau roman"~», Paris, \textit{Le Figaro Littéraire}, 1963/12/05-11, p.~1-26\par
	
		\vspace*{2cm}
		\setlength{\parindent}{0cm}
{\large\textsc{Œuvres d'Alain Robbe-Grillet citées}}
		\vspace*{1cm}
		\setlength{\parindent}{25pt}
		
		
		

		
		\textsc{Resnais}~Alain, \textit{L'Année dernière à Marienbad}, Silverfilms, Argos Films, Cinétel, Les Films Tamara, Precitel, Société Nouvelle des films Cormoran, Cineriz, Como Films, Terra Film Produktion, 1961\par 
	\textsc{Robbe-Grillet}~Alain, \textit{Les Gommes}, Paris, Les Éditions de Minuit, 1998 [1953]\par 
	\textsc{Robbe-Grillet}~Alain, \textit{Le Voyeur}, Paris, Les Éditions de Minuit,  [1955]\par 
	\textsc{Robbe-Grillet}~Alain, \textit{La Jalousie}, Paris, Les Éditions de Minuit, coll.~«~Double~», 1957 [2017]\par 
	\textsc{Robbe-Grillet}~Alain, \textit{Dans le labyrinthe}, Paris, Les Éditions de Minuit,  [1959]\par 
	\textsc{Robbe-Grillet}~Alain, \textit{L'Immortelle}, Cocinor, Como Films, Dino De Laurentiis Cinematografica, Les Films Tamara, 1963\par 
	
		\vspace*{2cm}
		\setlength{\parindent}{0cm}
{\large\textsc{Œuvres d'Alain Robbe-Grillet non citées}}
		\vspace*{1cm}
		\setlength{\parindent}{25pt}
		
		
		

		
		\textsc{Robbe-Grillet}~Alain, \textit{Djinn}, Paris, Les Éditions de Minuit, coll.~«~Double~», 1981 [2018]\par 
	\textsc{Robbe-Grillet}~Alain, \textit{Un Régicide}, Paris, Les Éditions de Minuit, 2006 [1978]\par 
	
		\vspace*{2cm}
		\setlength{\parindent}{0cm}
{\large\textsc{Publications ayant participées au débat}}
		\vspace*{1cm}
		\setlength{\parindent}{25pt}
		
		
		

		
		\textsc{Barthes}~Roland, «~Littérature objective~», Paris, \textit{Critique}, [vol.~XV, n°~86-87, juillet-août 1954], p.~581-591\par
	\textsc{Barthes}~Roland, «~Littérature littérale~», Paris, \textit{Critique}, [septembre-octobre~1955], p.~820-826\par
	\textsc{Henriot}~Émile, «~Le prix des critiques "Le Voyeur", d'Alain Robbe-Grillet~», Paris, \textit{Le Monde}, 15~juin~1955
			
			En ligne~: \hyperlink{https://www.lemonde.fr/archives/article/1955/06/15/le-prix-des-critiques-le-voyeur-d-alain-robbe-grillet\_1958094\_1819218.html}{https://www.lemonde.fr/archives/article/1955/06/15/le-prix-des-critiques-le-voyeur-d-alain-robbe-grillet\_1958094\_1819218.html}, consulté le \today
		\par
	\textsc{Huguenin}~Jean-René, «~Le nouveau roman~: une mode qui passe~», Paris, \textit{Arts}, [, n°~836, 27~septembre-03~ocotobre~1961], p.~1\par
	\textsc{Mauriac}~François, «~Technique du cageot~», Paris, \textit{Le Figaro littéraire}, 28~juillet~1956, p.~1-3\par
	\textsc{Ricardou}~Jean, \textit{Le Nouveau Roman suivi de Les Raisons de l'ensemble}, Paris, Éditions du Seuil, coll.~«~Points~», 1990 [1973]\par 
	\textsc{Rousseaux}~André, «~Les Surfaces d'Alain Robbe-Grillet~», Paris, \textit{Le Figaro littériare}, 13~avril~1957, p.~2\par
	\textsc{Sarraute}~Nathalie, \textit{L'Ère du soupçon}, Paris, Gallimard, coll.~«~Folio essais~», 2019 [1956]\par 
	\textsc{Simon}~Claude, \textsc{Veinstein}~Alain, \textit{La Nuit sur un plateau}, France culture, émission du 8~février~1988
			
			En ligne~: \hyperlink{https://www.radiofrance.fr/franceculture/podcasts/les-nuits-de-france-culture/claude-simon-j-ai-appris-a-ecrire-dans-joyce-et-dans-faulkner-6832681}{https://www.radiofrance.fr/franceculture/podcasts/les-nuits-de-france-culture/claude-simon-j-ai-appris-a-ecrire-dans-joyce-et-dans-faulkner-6832681}, consulté le \today
		\par 
	
		\vspace*{2cm}
		\setlength{\parindent}{0cm}
{\large\textsc{Sur le Nouveau Roman}}
		\vspace*{1cm}
		\setlength{\parindent}{25pt}
		
		
		

		
		\textsc{Allemand}~Roger-Michel, «~Le Temps de l'effacement~»,  \textit{Roman 20-50}, Hors-série n°~6, septembre~2010, p.~5-20
			
			En ligne~: \hyperlink{https://www.cairn.info/revue-roman2050-2010-3-page-5.htm\&wt.src=pdf}{https://www.cairn.info/revue-roman2050-2010-3-page-5.htm\&wt.src=pdf}, consulté le \today
		\par
	\textsc{Allemand}~Roger-Michelle, «~Nouveau Roman, Nouveau Monde~», Paris, \textit{Analyses}, vol.~8, n°~3, automne~2013, p.~142-170\par
	\textsc{Barilli}~Renato, «~Quand la jalousie s'attache aux choses~»,  \textit{Roman 20-50}, Hors-série n°~6, septembre~2010, p.~123-130
			
			En ligne~: \hyperlink{https://www.cairn.info/revue-roman2050-2010-3-page-123.htm\&wt.src=pdf}{https://www.cairn.info/revue-roman2050-2010-3-page-123.htm\&wt.src=pdf}, consulté le \today
		\par
	\textsc{Bishop}~Tom, \textsc{Jost}~François, \textsc{Pinget}~Robert, \textsc{Robbe-Grillet}~Alain, \textsc{Rybalka}~Michel, \textsc{Sarraute}~Nathalie, \textsc{Simon}~Claude, \textsc{Wittig}~Monique, «~Table ronde. Le Nouveau Roman : passé, présent,
				futur~»,  \textit{Cahiers Claude Simon}, 30~septembre~2019, p.~37-54\par
	\textsc{Boschetti}~Anna, \textsc{Sapiro}~Gisèle (dir.~), «~La recomposition de l'espace intellectuel en Europe après 1945~», Paris, La Découverte, \textit{L'Espace intellectuel en Europe}, 2009, p.~147-182

En ligne~: \href{https://doi.org/10.3917/dec.sapir.2009.01.0147}{https://doi.org/10.3917/dec.sapir.2009.01.0147}, consulté le \today\par
	\textsc{Butor} Michel, \textsc{Calle-Gruber} Mireille (dir.~), \textit{Œuvres complètes} vol.~2, Paris, Édition de la Différence, 2006\par
    \textsc{de Chalonge}~Florence, «~Les Défis du Nouveau roman~», Paris, \textit{Canopé}, 11/07/2017, p.~64-67\par
	\textsc{Denis}~Benoît, \textsc{Boujou}~Emmanuel (dir.), «~Engagement littéraire et morale de la littérature~», Presses Universitaires de Rennes, Rennes, \textit{L'Engagement littéraire}, 2005, p.~31-42
			
			En ligne~: \hyperlink{https://doi.org/10.4000/books.pur.30038}{https://doi.org/10.4000/books.pur.30038}, consulté le \today
		\par
	\textsc{Dosse}~François, «~Le jour où... S'inventa le «~nouveau roman~»~»,  \textit{Sciences Humaines}, n°~56, 2023/3, p.~55-54
			
			En ligne~: \hyperlink{https://doi.org/10.3917/sh.356.0054}{https://doi.org/10.3917/sh.356.0054}, consulté le \today
		\par
	\textsc{Duval}~Romain, «~Le formalisme contre les formes~», Paris, \textit{Nouvelle revue d’esthétique}, 2012/2, p.~141-151
			
			En ligne~: \hyperlink{https://www.cairn.info/revue-nouvelle-revue-d-esthetique-2012-2-page-141.htm}{https://www.cairn.info/revue-nouvelle-revue-d-esthetique-2012-2-page-141.htm}, consulté le \today
		\par
	\textsc{Genette}~Gérard, «~Vertige fixé~», \textit{Figures I}, Paris, Éditions du seuil, coll.~«~points essai~», 2014[1966], p.~69-90\par
	\textsc{Houppermans}~Sjef, «~Les Gommes~: un roman policier poli et scié~»,  \textit{Roman 20-50}, Hors-série n°~6, septembre~2010, p.~97-110
			
			En ligne~: \hyperlink{https://www.cairn.info/revue-roman2050-2010-3-page-97.htm\&wt.src=pdf}{https://www.cairn.info/revue-roman2050-2010-3-page-97.htm\&wt.src=pdf}, consulté le \today
		\par
	\textsc{Janvier}~Ludovic, «~Alain Robbe-Grillet et le couple fascination-liberté~», Paris, Les Éditions de Minuit, \textit{Une Parole exigeante}, 1964, p.~111-145

En ligne~: https://www.cairn.info/une-parole-exigeante--9782707335289-page-111.htm; consulté le \today \par
	\textsc{Kremer}~Nathalie, «~L'Espace du récit~»,  \textit{Poétique}, n°~192, 2022/2, p.~83-96
			
			En ligne~: \hyperlink{https://doi.org/10.3917/poeti.192.0083}{https://doi.org/10.3917/poeti.192.0083}, consulté le \today
		\par
	\textsc{Lambert}~Emmanuelle, «~Alaine~Robbe-Grillet et ses archives~»,  \textit{Société \& représentations}, n°~19, 2005/1, p.~197-210\par
	\textsc{Matvejevitch}~Predrag, «~L'engagement en littérature : vu sous les aspects de la sociologie et de la créatio~»,  \textit{L'Homme et la société}, n°~26, 1972, p.~119-132
			
			En ligne~: \hyperlink{http://dx.doi.org/10.3406/homso.1972.1725}{http://dx.doi.org/10.3406/homso.1972.1725}, consulté le \today
		\par
	\textsc{Migeot}~François, «~Inconscient et poétique~: La Jalousie~»,  \textit{Roman 20-50}, Hors-série n°~6, septembre~2010, p.~177-194
			
			En ligne~: \hyperlink{https://www.cairn.info/revue-roman2050-2010-3-page-177.htm\&wt.src=pdf}{https://www.cairn.info/revue-roman2050-2010-3-page-177.htm\&wt.src=pdf}, consulté le \today
		\par
	\textsc{Milat}~Christian, «~Le Fils jaloux des Gommes, narrateur «~absent~»de La Jalousie~?~»,  \textit{Roman 20-50}, Hors-série n°~6, septembre~2010, p.~21-34
			
			En ligne~: \hyperlink{https://www.cairn.info/revue-roman2050-2010-3-page-21.htm\&wt.src=pdf}{https://www.cairn.info/revue-roman2050-2010-3-page-21.htm\&wt.src=pdf}, consulté le \today
		\par
	\textsc{Milat}~Christian, «~Sartre et Robbe-Grillet, ou les chemins de l'écriture~», Paris, \textit{Revue d'Histoire littéraire de la France}, Janvier-Février, 2002, p.~83-96
			
			En ligne~: \hyperlink{https://www.jstor.org/stable/40534639}{https://www.jstor.org/stable/40534639}, consulté le \today
		\par
	\textsc{Morissette}~Bruce, \textit{Les romans de Robbe-Grillet}, Paris, Les Éditions de Minuit, 1971 [1963]\par 
	\textsc{Simonin}~Anne, «~La Littérature saisie par l'histoire. Nouveau Roman et guerre d'Algérie aux Éditions de Minuit~»,  \textit{Actes de la recherche en sciences sociales}, vol.~111-112, mars~1996, p.~59-75
			
			En ligne~: \hyperlink{https://doi.org/10.3406/arss.1996.3168}{https://doi.org/10.3406/arss.1996.3168}, consulté le \today
		\par
	\textsc{Simonin}~Anne, «~Le catalogue de l'éditeur, un outil pour l'histoire. L'exemple des éditions de minuit~»,  \textit{Vingtième Siècle. Revue d'histoire}, n°~81, 2004/1, p.~119-129
			
			En ligne~: \hyperlink{https://doi.org/10.3917/ving.081.0119}{https://doi.org/10.3917/ving.081.0119}, consulté le \today
		\par
	\textsc{Yanoshevsky}~Galia, «~La critique littéraire et les romans de Robbe-Grillet~», Lille, \textit{Roman 20-50}, 2010/3 (hors série n°~6), p.~67-82
			
			En ligne~: \hyperlink{https://www.cairn.info/revue-roman2050-2010-3-page-67.htm}{https://www.cairn.info/revue-roman2050-2010-3-page-67.htm}, consulté le \today
		\par
	\textsc{Yanoshevsky}~Galia, \textit{Les discours du Nouveau Roman : Essais, entretiens, débats}, Villeneuve d'Ascq, Presses universitaires du Septentrion, 2006
			
			En ligne~: \hyperlink{https://doi-org.ressources-electroniques.univ-lille.fr/10.4000/books.septentrion.54734}{https://doi-org.ressources-electroniques.univ-lille.fr/10.4000/books.septentrion.54734}, consulté le \today
		\par 
	
		\vspace*{2cm}
		\setlength{\parindent}{0cm}
{\large\textsc{Œuvres citées}}
		\vspace*{1cm}
		\setlength{\parindent}{25pt}
		
		
		

		
		\textsc{Balzac}~Honoré, \textit{Le Père Goriot}, Paris, Pocket, coll.~«~Pocket Classique~», 1989 [1842]\par 
	\textsc{Balzac}~Honoré, \textit{Le Médecin de campagne}, Paris, Gallimard, coll.~«~folio classique~», 2021 [1833]\par 
	\textsc{Beckett}~Samuel, \textit{Molloy}, Paris, Les Éditions de Minuit, coll.~«~Double~», 1982 [1951]\par 
	\textsc{Beckett}~Samuel, \textit{Fin de partie}, Paris, Les Éditions de Minuit, 2009 [1957]\par 
	\textsc{Beckett}~Samuel, \textit{En attendant Godot}, Paris, Les Éditions de Minuit, 2007 [1952]\par 
	\textsc{Camus}~Albert, \textit{L'Étranger}, Paris, Gallimard, coll.~«~Folio~», 2010 [1942]\par 
	\textsc{Céline}~, \textit{Voyage au bout de la nuit}, Paris, Gallimard, coll.~«~Folioplus classiques~», 2011 [1952]\par 
	\textsc{Dostoïevski}~Fédor, \textsc{Markowicz} André (trad.), \textit{Les Fères Karamazov}, Paris, Actes Sud, coll.~«~Babel~», 2002 pour la traduction française,  [1880]\par 
	\textsc{Faulkner}~William, \textsc{Coindreau} Maurice Edgar (trad.), \textit{Le Bruit et la fureur}, Paris, Gallimard, coll.~«~Folio~», 20071972 pour la traduction française,  [1931]\par 
	\textsc{Flaubert}~Gustave, \textit{Madame Bovary}, Paris, Pocket, coll.~«~Pocket Classique~», 1998 [1857]\par 
	\textsc{Hugo}~Victor, \textit{Ruy Blas}, Paris, Livre de poche, 2009 [1838]\par 
	\textsc{Kafka}~Franz, \textsc{Nesme} Axel (trad.), \textit{Le Château}, Paris, Librairie Générale Française, coll.~«~Le Livre de poche~», 20072001 pour la traduction française,  [1926]\par 
	\textsc{Kafka}~Franz, \textsc{Vialatte} Alexandre (trad.), \textit{Le Procès}, Paris, Gallimard, coll.~«~Folio classique~», 20091933 pour la traduction française,  [1925]\par 
	\textsc{Pascal}~Blaise, \textit{Pensées}, Paris, Pocket, coll.~«~Agora~», 2009 [1670]\par 
	\textsc{Pinget}~Robert, \textit{Mahu ou le matériau}, Paris, Les Éditions de Minuit, 1997 [1952]\par 
	\textsc{Pinget}~Robert, \textit{Le Renard et la boussole}, Paris, Les Éditions de Minuit, 2000 [1953]\par 
	\textsc{Ponge}~Francis, \textit{Le Parti pris des choses}, Paris, Gallimard, coll.~«~nrf~», 2013 [1942]\par 
	\textsc{Proust}~Marcel, \textit{Du côté de chez Swann}, Paris, Gallimard, coll.~«~Folio classique~», 1988 [1913]\par 
	\textsc{Queneau}~Raymond, \textit{Le Chiendent}, Paris, Gallimard, coll.~«~Folio~», 2009 [1933]\par 
	\textsc{Roussel}~Raymond, \textit{Locus solus}, Paris, Flammarion, coll.~«~GF~», 2005 [1965]\par 
	\textsc{Sartre}~Jean-Paul, \textit{La Nausée}, Paris, Gallimard, coll.~«~Folio~», 2009 [1938]\par 
	\textsc{Sartre}~Jean-Paul, \textit{Qu'est-ce que la littérature ?}, Paris, Gallimard, coll.~«~Folio essais~», 2008 [1948]\par 
	\textsc{Sartre}~Jean-Paul, \textit{Situations, I Critiques littéraires}, Paris, Gallimard, coll.~«~Folio essais~», 2020 [1947]\par 
	\textsc{Stendhal}~, \textit{La Chartreuse de Parme}, Paris, Pocket, coll.~«~Pocket Classiques~», 2008 [1839]\par 
	\textsc{Svevo}~Italo, \textsc{Michel Fusco} Paul-Henri Mario (trad.), \textit{La Conscience de Zeno}, Paris, Gallimard, coll.~«~Folio~», 20101986 pour la traduction française,  [1954]\par 
	
		\vspace*{2cm}
		\setlength{\parindent}{0cm}
{\large\textsc{Modèle d'édition critique}}
		\vspace*{1cm}
		\setlength{\parindent}{25pt}
		
		
		

		
		
		\textit{Site du fonds Jean Ricardou} en ligne~: \hyperlink{https://jeanricardou.org/}{https://jeanricardou.org/}, consulté le 23 novembre 2022\par

	\textsc{Duras}~Marguerite, \textit{Œuvres complètes}, Paris, Gallimard, coll.~«~Pléiade~», 2011\par 
	\textsc{Valincourt}, \textit{Lettres à Madame la marquise *** sur le sujet de La Princese de Clèves}, Paris, Flammarion, coll.~«~Garnier Flammarion~», 2021 [1678]\par